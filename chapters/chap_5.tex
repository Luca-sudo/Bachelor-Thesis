\chapter{Adding Variables}

Before continuing, let us examine the limitations of the \nameref{def:prog-lang-4} from the previous chapter. We are not able to declare variables, control flow, or constants. All of those features will be implemented in this chapter. In order to declare and use variables in our programming language, we need to provide a definition of an environment, which comprises the concrete values of variables. As we will see, introducing variables will demand cautious reformulation of most evaluation rules to account for environments.

Similarly to environments, we will introduce contexts for the type rules. While an environment maps variables to concrete values, a context will map variables to types. This will also require carefully embellishing previous type rules, to provide rules that are rigorous with respect to contexts. We define the syntax of the augmented programming language below:


\begin{definition}[Programming Language]
   \label{def:prog-lang-5}

\begin{align*}
   e := ~~~ & \letexp{x}{e_1}{e_2}		& \text{(let)}\\
            & \tick{k}				& \text{(tick)}\\
	    & x					& \text{(var)}\\
	    & \true ~~| ~~\false		& \text{(bool constructors)}\\
	    & q					& \text{(const rational)}\\
\end{align*}

\end{definition}


\section{Evaluation Semantics}

We begin this section by introducing environments, which provide a necessary link between variable names and values. For this, we define an \nameref{def:environment}, which is a mapping from variable names to values. Along with this, we introduce two notations related to environments, one for retrieving the value of the variable and another notation that allows us to check if an environmant contains a specific variable name; This will be needed to ensure that rules featuring variables are correctly defined.

\begin{definition}[Environment]\label{def:environment}
   We define an \textbf{environment} \(E: \text{\Var} \partialto \text{\Vals}\) as a partial function from variable names (\Var) to values (\Vals). 
\end{definition}

\begin{definition}\label{def:environment-getter}	
   Let \(E\) be an \nameref{def:environment} and \(x\) be a variable name, we then write \(E(x)\) for the value of \(x\) in the environment \(E\):

   \[
      E(x) =
      \begin{cases}
	 E(x)		    & \mbox{,if } \contains{E}{x} \\
	 \textbf{undefined} &\mbox{,else}
      \end{cases}
   \]
\end{definition}

\begin{definition}\label{def:in-environment}
   Given an \nameref{def:environment} \(E\) and a variable name \(x\), we define \(\contains{E}{x}\) as follows:
   \[
      \contains{E}{x} = 
      \begin{cases*}
	 \false & \mbox{,if E(x) is undefined}\\
	 \true & \mbox{,else} 
      \end{cases*}
   \]
\end{definition}

Here, we define a partial function \(f\) as a function either maps a value \(x\) to a value in the codomain or returns \textbf{undefined} otherwise. Defining environments using partial functions precisely captures the notion of an environment, while providing us with the needed flexibility; Once new variables are instantiated, we are able to define their value in the environment pointwise:

\begin{definition}\label{def:environment-augment}	
   Let \(E\) be an \nameref{def:environment}, \(a\) be a variable name and \(b\) be the value assigned to \(a\). We write \(E[x \mapsto y]\) for the environment \(E\), where \(a\) is additionally mapped onto \(b\):

   \[
      \envaugment{E}{a}{b} = 
      \begin{cases}
	 E(x)  & \mbox{,if } x \neq a\\
	 b     & \mbox{,else } 
      \end{cases}
   \]

\end{definition}

As we now evaluate expressions with respect to an \nameref{def:environment}, we need to update \cref{def:evaluation-judgement} to accommodate environments:

\begin{definition}[Evaluation Judgement with Environments]\label{def:eval-judgement-environments}
   Let \(E\) be an \nameref{def:environment}, \(e\) an expression of our programming language (\cref{def:prog-lang-5}), \(v\) the value obtained from evaluating \(e\), and let \((p_0, p_1)\) be a \nameref{def:resource-pair}. We then write the following for the evaluation judgement that incorporates an environment:

   \[
      \evals{E}{e}{v}{p_0}{p_1}
   \]
	
\end{definition}

Having all the necessary definitions at hand, let us now define evaluation rules for our programming language. We start with the rule \((E:var)\), which provides an explicit handle for resolving variable names to their values. We want to highlight that (E:var) has no cost associated with it. This can simply be changed by adjusting the a priori resources for the rule:

\[
   \inference[(E:var)]
   {\contains{E}{x}}
   {\evals{E}{x}{E(x)}{}{}}
\]

Next, we define one rule (E:bool) that unifies the evaluation of both boolean constants. Again, we decide to associate no cost to the evaluation of boolean values, though for a different reason this time. We want the programmer to have the highest degree of jurisdiction, as a result, we want \(\tick\) to be the prominent means of administering cost. The same line of thought holds for the creation of numeric constants.

\[
   \inference[(E:bool)]{b \in \{\true, \false\}}{\evals{E}{b}{b}{}{}}%
   \quad\quad\quad
   \inference[(E:int)]{n \in \mathbb{Z}}{\evals{E}{n}{n}{}{}}
\]

Next, we introduce environments into the evaluation rule for \(\tick{}\). It is worth noting that the evaluation of a \(\tick{}\) expression is \emph{invariant} under the environment, as it does not rely on any state.

\[
   \inference[(E:tick)]
   {}
   {\evals{E}{\tick{q}}{()}{a}{b}}%
   \qquad 
   \begin{aligned}
      \text{where }  &a = \max(q, 0)\\
      \text{and }    &b = \max(0, -q) 
   \end{aligned}
\]

Given an expression \(\tick{k}\), there are two mutually exclusive cases: \(k\) is positive, or it is negative. In the former case, \(a\) reduces to \(k\) and \(b\) to zero. In the latter case, \(a\) is zero and \(b\) reduces to \(-q\), which is again a positive number - as desired. Splitting this evaluation rule up into two distinct rules for positive and negative \(k\) is equally valid.

The last rule missing is (E:let). Here, we need to carefully weave the value that is the result of the first expression into the second expression. Note that the values of \(a\) and \(b\) are closely linked to \cref{def:multiplying-pairs}, corresponding to the first and second entry of the tuple, respectively.

\[
   \inference[(E:let)]
   {\evals{E}{e_1}{v_1}{p_0}{p_1} \qquad \evals{\envaugment{E}{x}{v_1}}{e_2}{v_2}{q_0}{q_1}}
   {\evals{E}{\letexp{x}{e_1}{e_2}}{v_2}{a}{b}}%
   \qquad
   \begin{aligned}
      \text{where }  &a = p_0 - p_1 + \max(p_1, q_0)\\
      \text{and }    &b = q_1 - q_0 + \max(p_1, q_0) 
   \end{aligned}
\]

\section{Type Rules}\label{sec:type-rules-5}

In this section, we will first motivate the new type system, introduce \nameref{def:context}s and define the necessary type rules.With respect to the type system, we are required to introduce two new types: \bool~and \typeint. This yields the following type system:

\begin{figure}[H]
   \[A := \unit~|~\text{\bool}~|~\text{\typeint}\]
   \caption{Type system with booleans and integers added.}
   \label{fig:type-system-5}
\end{figure}

As as result of extending the type system, \cref{def:type-product} becomes ill-defined; It is not clear what the result of \(\ratype{\mbox{\typeint}}{p_0}{p_1} \ltimes \ratype{\textrm{\bool}}{q_0}{q_1}\) would be. As this type signature describes a sequence of two operations, where the first operation yields a value of \typeint, which is consumed by the second operation yielding a value of type \bool as the final result. This motivates the following, updated, definition:

\begin{definition}
   Let \(A\) and \(B\) be types from \cref{fig:type-system-5}, with the respective resource pairs \((p_0, p_1)\) and \((q_0, q_1)\). We then define \(\ratype{A}{p_0}{p_1} \ltimes \ratype{B}{q_0}{q_1} = \ratype{B}{p_0 - p_1 + \max(p_1, q_0)}{q_1 - q_0 + \max(p_1, q_0)}\).
\end{definition}

Next, we introduce \nameref{def:context}s. Similar to \nameref{def:environment}s, a context is a partial function that assigns to variable names a type from \cref{fig:type-system-5}:

\begin{definition}[Context]\label{def:context}
   We define a \emph{context} \(\Gamma : \text{\Var} \partialto \text{\Types}\) as a partial function from variable names to types.
\end{definition}

We know define some notations that are tightly linked to the ones introduced in \cref{def:in-environment}, \cref{def:environment-getter} and \cref{def:environment-augment}:


\begin{definition}\label{def:context-getter}
   Let \(\Gamma\) be a \emph{context} and \(x\) be a variable name, then \(\Gamma(x)\) returns either the type of \(x\) or undefined.
\end{definition}

\begin{definition}\label{def:in-context}
   Let \(\Gamma\) be a \emph{context} and \(x\) be a variable name, we define \(\contains{\Gamma}{x}\) as follows:
   \[
      x \in \Gamma =
      \begin{cases}
	 \false	     & \text{,if } \Gamma(x) \text{ is undefined}\\
	 \true	     & \text{,else}
      \end{cases}
   \]
\end{definition}

\begin{definition}\label{def:context-augment}
   Let \(\Gamma\) be a context, \(A\) be a type and let \(x : A\) be a variable. We can embed the variable \(x\) with its associated type \(A\) by writing \(\Gamma ; x:A\).
\end{definition}

\begin{definition}[Typing Judgement with contexts]\label{def:type-judgement-context}
   Let \(\Gamma\) be a \nameref{def:context}, \(e\) an expression of \cref{def:prog-lang-5} with the return type \(A\), and let \((p_0, p_1)\) be a \nameref{def:resource-pair}. We then denote a typing judgement with respect to the context \(\Gamma\) by writing:
   \[
      \typing{\Gamma}{e}{\ratype{A}{p_0}{p_1}}
   \]
\end{definition}

This comprises most definitions that are needed to define the necessary type rules. We will introduce further notations, once they become necessary, clearly motivating their use and definition. Let us start by introducing the rule (T:var), which allows us to add a variable and its associated type into a context.

\[
   \inference[(T:var)]
   {}
   {\typing{\Gamma; x: A}{x}{A}}
\]

Next, we introduce two rules, one for boolean constants and one for integer constands. The rule (T:bool) assigns a type \bool~to any occurence of \(\true\) or \(\false\), and, similarly, the rule (T:int) assigns a type \typeint to any occurence of integer constants.

\[
   \inference[(T:bool)]
   {b \in \{\true, \false\}}
   {\typing{\emptycontext}{b}{\text{\bool}}}%
   \qquad
   \inference[(T:int)]
   {n \in \mathbb{Z}}
   {\typing{\emptycontext}{n}{\text{\typeint}}}
\]

Most significantly, we used the \emph{empty context}, written as \(\emptycontext\). This is the context that contains no information. We do this for two reasons: Firstly, the rule is more declarative as a result, because the typing is \emph{invariant} with respect to the context - an integer is always an integer. Secondly, this is precisely the smallest context for which the rule holds true; As such, the empty context forms the most general case for these rules to hold. Now, we are even able to type constant integers or booleans with an arbitrarily large \nameref{def:context} by applying the (T:var) rule, such that the contexts match. 

Defining the type rule for the \(\tick{}\) is straightforward, we add a context \(\Gamma\) to the rule from the previous chapter \cref{def:tr-tick-4}:

\[
   \inference[(T:tick)]
   {q_0 \geq k + q_1}
   {\typing{\emptycontext}{\tick{k}}{\ratype{\unit}{q_0}{q_1}}}
\]

For the let expression, we provide a definition that differs from the one provided in \cref{def:tr-let-4}. For this, we make the rule (T:let) stricter, by requiring that the \nameref{def:resource-disparity} between the two expressions is zero. As this would deny typing for \(\tick{}\) expressions that have a non-zero resource-disparity, we introduce another rule (T:relax) that allows us to \emph{relax} the resource-annotations into ones that permit applying (T:let) - only if that is possible. 

\[
   \inference[(T:relax)]
   {}
   {}
\]

As the subexpressions of the let expression might have different contexts, we need to account for that as well. To permit a consistent formulation of the type rule, we need to add an assumption: the two contexts need to be \emph{disjoint}. This will allow us to avoid any typing inconsistencies that could arise from mismatched types between the two contexts. 
For this, a variable \(x\) could be present in both contexts \(\Gamma_1\) and \(\Gamma_2\), with differnt types assigned to the variable:

\[
   \typing{\Gamma_1}{x}{A} \qquad \typing{\Gamma_2}{x}{B}
\]

This is prevented, by demanding that the contexts are \emph{disjoint}:

\[
   \inference[(T:let)]
   {\typing{\Gamma_1}{e_1}{\ratype{A}{p_0}{m}} \qquad \typing{\Gamma_2}{e_2}{\ratype{B}{m}{q_1}} \qquad \Gamma_1 \cap \Gamma_2 = \emptyset}
   {\typing{\Gamma_1; \Gamma_2}{\letexp{x}{e_1}{e_2}}{\ratype{B}{p_0}{q_1}}}
\]


















