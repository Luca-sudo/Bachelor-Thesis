\chapter{LetTick Language}

In this chapter, we introduce our first programming language. This language will form a core, so that subsequent chapters extend this language incrementally. To this end, we define the type system, along with the associated evaluation semantics and type rules. The chapter then concludes with a proof of soundness, showing that the relation between evaluation semantics and type rules is consistent - the resource bounds assigned to types indeed provide a lower bound on the concrete resource consumption of evaluating an expression.

In order to elicit resource bounds for programs, we need two things: (1) A construct in the programming language that allows altering the resource demand and (2) Types that reflect the resource demand. The first problem is solved by introducing the instruction \(\tick{k}\), where \(k\) is an integer, indicating the amount of resources that are either consumed of freed. The second problem is solved by embellishing the return type with a resource demand.

\section{Language Syntax}
We begin by defining the programming language and the associated type system. Initially, the syntax for let expressions may seem obfuscated. We choose this syntax with later chapters in mind, to make extension as easy as possible. Generally, the literal after \textbf{let} is used to bind the result of \(e_1\). For this chapter, we do not yet bind any literals - for now, we use \_ as a placeholder.

\begin{definition}[LetTick syntax]
   \label{def:prog-lang-4}

\begin{align*}
   e := ~~~ & \letexp{\_}{e_1}{e_2}             & \text{(let)}\\
            & \tick{q}                         & \text{(tick)}
\end{align*}
\end{definition}

Expressions inside the LetTick language are sequences of tick instructions. As an example, consider the programs below. In \cref{eqn:let-tick-derivation} we provide the associated type derivation and derive resource bounds for this program.

\begin{example} \label{ex:let-tick-prog}
   \begin{align*}
   \letexp{\_}{\tick{3}}{} \\
   \letexp{\_}{\tick{-2}}{} \\
   \tick{5}
   \end{align*}
\end{example}

Intuitively, we have three distinct steps. The first one consumes 3 resources, the second one frees up 2 resources and the last one consumes 5 resources. 

\begin{definition}[LetTick types]\label{fig:type-system}
   \[\type := \unit \]
\end{definition}

Note that the unit type has exactly one, unique value that we denote by \(\bullet\). A specific constructor for \(\bullet\) is not required, we can already generate this value. By writing \tick{0}, no resources are consumed and \(\bullet\) is returned. Thus, it serves as a constructor for \(\bullet\).


\section{Evaluation Semantics}

When evaluating an expression from \cref{def:prog-lang-4}, we are interested in (1) The value that the expression evaluates to and (2) The resource demand required for evaluation. Both are defined by the evaluation rules below. 

Before we can define evaluation rules, we need to introduce new notation. We denote by \(\evals{}{e}{v}{p_0}{p_1}\) the fact that the expression \(e\) evaluates to the value \(v\), while requiring \(p_0\) initial and \(p_1\) residual resources. The definition of the \emph{evaluation judgement} is then given by the evaluation rules.

First, we define a rule for \tick{k} instructions. Such an instruction either frees resources or consumes them, if k is negative or positive, respectively. We model these cases as two evaluation rules.

\[
   \inference[\((E:tick_{+})\)]
   {q \geq 0}
   {\evals{}{\tick{k}}{\bullet}{k}{0}}
   \qquad
   \inference[\((E:tick_{-})\)]
   {q < 0}
   {\evals{}{\tick{k}}{\bullet}{0}{|k|}}
\]

Next, we need a rule that allows us to combine two expression into a let expression. The resulting resource demand is the composition of the resource demand of each sub-expression. For a definition of the \(\sequence\) operator, please refer to \cref{def:multiplying-pairs}.

\[
   \inference[(E:let)]
   {\evals{}{e_1}{\bullet}{p_0}{p_1} ~~,~~ \evals{}{e_2}{\bullet}{q_0}{q_1}}
   {\evals{}{\letexp{}{e_1}{e_1}}{\bullet}{r_0}{r_1}}
   \qquad
   \begin{aligned}
      \text{where }  &(r_0, r_1) = (p_0, p_1) \sequence (q_0, q_1) 
   \end{aligned}
\]


\begin{definition}[Evaluation Judgement]\label{def:evaluation-judgement}
   Let \(e\) be an expression of our programming language, let \(v\) be the value that is obtained from evaluating \(e\). Furthermore, let \((p_0, p_1) \in \demand\). 
   Where \(p_0\) are the initial resources needed to evaluate \(e\), and \(p_1\) resources are freed afterward.
   
   \[
      \evals{}{e}{v}{p_0}{p_1}
   \]
\end{definition}

In the next section, we introduce type rules that are tightly linked to the evaluation semantics defined now. 

\section{Type System}
The evaluation judgement in the previous section defined our \emph{ground truth}; That is, using the rules defined there, we can exactly calculate the resource-consumption of an arbitrary expression from \cref{def:prog-lang-4}. The type rules introduced in this chapter will allow us to assign a type to an expression, where the resource-annotations of that type provide an \textbf{upper-bound} on the resource-consumption. As a result, type inference can be used to determine an upper bound on the resource consumption, without executing a program.

\subsection{Resource-aware Type System}

We start by introducing \emph{resource-aware types}, which comprise two parts: the type assigned to an expression, together with a resource-pair representing the inferred resource consumption. To this end, we provide an ergonomic notation for resource-aware types:

\begin{definition}[Resource-aware types]\label{def:ra-type}
   Let \(A \in \type\) and \((p_0, p_1) \in \demand\). We define resource-aware types as tuples \((A, (p_0, p_1)) \in \type \times \demand\). Most of the time we write \(\ratype{A}{p_0}{p_1}\) for \((A, (p_0, p_1))\). 
\end{definition}

For this chapter, the only type that \(A\) can adopt is \(\unit\). Therefore, resource-aware types have the form \(\ratype{\unit}{p_0}{p_1}\) \emph{for this chapter}.

\subsection{Inference Rules}

For let expressions, we compose resource demands using the sequencing operator \(\sequence\) introduced in \cref{def:multiplying-pairs}. Note that this rule matches its associated evaluation rule.

\[
   \label{def:tr-let-4}
   \vcenter{\hbox{
      \inference[(T:let)]
      {\typing{}{e_1}{\ratype{\unit}{p_0}{p_1}} \quad \typing{}{e_2}{\ratype{\unit}{q_0}{q_1}}}
      {\typing{}{\letexp{\_}{e_1}{e_2}}{\ratype{\unit}{r_0}{r_1}}}
}}\qquad \begin{aligned}
      (r_0, r_1) = (p_0, p_1) \sequence (q_0, q_1)
   \end{aligned}
\]

The expression \(\tick{k}\) consumes (or frees) \(k\) resources. We can type this expression with any resource demand \((q_0, q_1)\) where \(q_0 - q_1 \geq k\). Compared to the evaluation rules in the previous section, this rule is more generic - the resource demand \((p_0, p_1)\) is not unique.

Let us elaborate on why this makes sense. 

\[
   \label{def:tr-tick-4}
   \inference[(T:tick)]
   {q_0 \geq k + q_1}
   {\typing{}{\tick{k}}{\ratype{\unit}{q_0}{q_1}}}
\]


Let us now perform an exemplary type derivation using the type rules we just introduced, to get a better understanding of how the definitions orchestrate together. For this, consider the program from \cref{ex:let-tick-prog}. The resulting type derivation is displayed below.

\[
   \inference
   {
      \inference
      {q_0 \geq 3 + q_1}
      {\typing{}{\tick{3}}{\ratype{\unit}{q_0}{q_1}}}
       &
      \inference
      {
         \inference
         {t_0 \geq -2 + t_1}
         {\typing{}{\tick{-2}}{\ratype{\unit}{t_0}{t_1}}}
          &
         \inference
         {s_0 \geq 5 + s_1}
         {\typing{}{\tick{5}}{\ratype{\unit}{s_0}{s_1}}}
      }
      {\typing{}{\letexp{\_}{\tick{-2}}{\tick{5}}}{\ratype{\unit}{r_0}{r_1}}}
   }
   {\typing{}{\letexp{\_}{\tick{3}}{\letexp{\_}{\tick{-2}}{\tick{5}}}}{\ratype{\unit}{p_0}{p_1}}}
\]

We are now left with the task of resolving the collected, numeric constraints. These are the collected constraints:

\begin{align}
   q_0 & \geq 3 + q_1 \label{eqn:derivation-tick-1}\\
   t_0 & \geq t_1 - 2 \label{eqn:derivation-tick-2}\\
   s_0 & \geq 5 + s_1 \label{eqn:derivation-tick-3}\\
   (r_0, r_1) & = (t_0, t_1) \sequence (s_0, s_1) \label{eqn:derivation-seq-1}\\
   (p_0, p_1) & = (q_0, q_1) \sequence (r_0, r_1) \label{eqn:derivation-seq-2}
\end{align}

\cref{eqn:derivation-tick-1}, \cref{eqn:derivation-tick-2} and \cref{eqn:derivation-tick-3} stem from applications of (T:Tick). \cref{eqn:derivation-seq-1} and \cref{eqn:derivation-seq-2} stem from applications of (T:Let). To derive a resulting resource demand, we must find a solution to the set of constraints above. For this specific instance, a valid solution would be:

\begin{align*}
   q_0 &= 3 & q_1 & = 0\\
   t_0 &= 0 & t_1 & = 2 \\
   s_0 &= 5 & s_1 & = 0
\end{align*}

Thus, we can calculate \((p_0, p_1)\) using \cref{eqn:derivation-seq-1} and \cref{eqn:derivation-seq-2}. We get:

\begin{align*}
   (r_0, r_1) & = (0, 2) \sequence (5, 0) = (3, 0) \\
   (p_0, p_1) & = (3, 0) \sequence (3, 0) = (6, 0)
\end{align*}

As a visual aid, consider the associated resource diagram: 

\begin{align*}
   \tikzfig{ex351} & \implies & \tikzfig{ex352}
\end{align*}

This completes our type derivation, as inserting the calculated resource-annotations yields our desired type.
\[
   \letexp{\_}{\tick{3}}{\letexp{\_}{\tick{-2}}{\tick{5}}}~:~\ratype{\unit}{6}{0}
\]

Having a solid grasp of how the type rules orchestrate together, we now prove soundness for the LetTick programming language.


\section{Soundness}
Ultimately, we want our type rules to be consistent with the operational semantics defined previously. This will allow us to reason about the type of a program, while ensuring that the resource-annotations of the resulting type line up with the concrete resource consumption of the expression.

Let us first sketch the structure of our soundness proof:

If \(\typing{}{e}{\ratype{\unit}{p_0}{p_1}}\) and \(\evals{}{e}{()}{q_0}{q_1}\), then the following hold true:
The result \(()\) of the expression \(e\) has a type that is consistent with the one assigned to \(e\). The initial resources of the type are not exceeded in the concrete evaluation of the expression \(e\). The resource consumption assigned to the type is not exceeded by the evaluation, either. We can state the same more precisely, by using the relaxation relation defined in \cref{def:relaxation}.

\begin{theorem}[Soundness of typing for LetTick language]\label{thm:soundness-let-tick}
   Let e be an expression in our programming language, and let \((q_0, q_1), (p_0, p_1) \in \resource\). 
   \center If \(\typing{}{e}{\ratype{\unit}{p_0}{p_1}}\) and \(\evals{}{e}{()}{q_0}{q_1}\), then \((p_0, p_1) \relaxation (q_0, q_1)\).
\end{theorem}

\begin{proof}
In the first part, I prove that the inequality \(p_0 \geq q_0\) holds. We use structural induction on the type derivation of the expression, which concretely means, we assume that the expression \(e\) has the type \(\typing{}{e}{\ratype{\unit}{p_0}{p_1}}\) and that the expression evaluates, like so \(\evals{}{e}{()}{q_0}{q_1}\). We now need to prove that, for every type rule that could be the last rule applied in a type derivation, the inequality holds.

   (tick): We make a case distinction. Let \(q \geq 0\). Then the following hold by assumption:
   \begin{align}
      q     &\geq 0 \label{4.1}\\
      q_0   &= q \label{4.2}\\
      q_1   &= 0 \label{4.3}\\
      p_0   &\geq q + p_1 \label{4.4}
   \end{align}
   Since \(p_1 \geq 0\) by definition, using \ref{4.4}, \(p_0 \geq q\) holds. Applying this inequality to \ref{4.2}, we get that \(p_0 \geq q_0\). Which proves the first part.
   Next, we show \(p_0 - p_1 \geq q_0 - q_1\). Combining \ref{4.4} with \ref{4.2}, we get \(p_0 \geq q_0 + p_1\). Subtracting \(p_1\) from both sides, we get \(p_0 - p_1 \geq q_0\). Since \(q_1 \geq 0\) by definition as well, it especially holds that \(p_0 - p_1 \geq q_0 - q_1\). Which concludes the proof for the case of \(q \geq 0\).
   For the second case, we assume that \(q < 0\). Then, the following hold:
   \begin{align}
      q_1   &= q \label{4.5}\\
      q_0   &= 0 \label{4.6}
   \end{align}
   Furthermore, \ref{4.1} and \ref{4.4} also hold true. By definition \(p_0 \geq 0\) holds. Using \ref{4.6}, we get \(p_0 \geq q_0\). 
   Let us now prove soundness with respect to resource consumption. Subtracting \(p_1\) from both sides in \ref{4.4}, we get \(p_0 - p_1 \geq q \). Applying \cref{4.5}, we get \(p_0 - p_1 \geq q_1\). Because \(q_1 \geq 0\) by definition, it also holds that \(p_0 - p_1 \geq 0 - q_1\). Using \ref{4.6}, we get \(p_0 - p_1 \geq q_0 - q_1\), concluding our proof for tick.

   (let):  If the let rule was applied, the expression has the form \(\letexp{\_}{e_1}{e_2}\). Our induction hypothesis includes judgements, and soundness, for the sub-expressions \(e_1, e_2\). We list all assumptions below:
   \begin{align*}
      &\typing{}{e_1}{\ratype{\unit}{p_0}{p_1}}                   & &(p_0, p_1)\relaxation (m_0, m_1)\\
      &\evals{}{e_1}{v_1}{m_0}{m_1}                               & &(q_0, q_1)\relaxation (n_0, n_1)\\
      &\typing{}{e_2}{\ratype{\unit}{q_0}{q_1}}                   & &(r_0, r_1) = (p_0, p_1) \sequence (q_0, q_1)\\
      &\evals{}{e_2}{v_2}{n_0}{n_1}                               & &(s_0, s_1) = (m_0, m_1) \sequence (n_0, n_1)\\
      &\typing{}{\letexp{\_}{e_1}{e_2}}{\ratype{\unit}{r_0}{r_1}} & &\evals{}{\letexp{\_}{e_1}{e_2}}{v_2}{s_0}{s_1}
   \end{align*}
   
   To prove soundness for the let expression, we need to prove \((r_0, r_1) \relaxation (s_0, s_1)\). We do this by proving soundness for two mutually exclusive conditions: \(n_0 \geq m_1\) and \(n_0 < m_1\). Before introducing the case distinction, we prove soundness for the resource consumption in the general case:
   \begin{align}
      s_0 - s_1   &= m_0 - m_1 + \max(m_1, s_0) - (n_1 - n_0 + \max(m_1, s_0)) \\
                  &= m_0 - m_1 + n_0 - n_1 + \max(m_1, s_0) - \max(m_1, s_0) \\
                  &= m_0 - m_1 + n_0 - n_1 \\
                  &\leq p_0 - p_1 + q_0 - q_1 \label{eqn:3.16}\\
                  &= p_0 - p_1 + \max(p_1, q_0) + q_0 - q_1 - \max(p_1, q_0) \\
                  &= \underbrace{p_0 - p_1 + \max(p_1, q_0)}_{= r_0} - \underbrace{(q_1 - q_0 + \max(p_1, q_0))}_{= r_1}
   \end{align}

   Where \cref{eqn:3.16} uses \((p_0, p_1) \relaxation (m_0, m_1), (q_0, q_1) \relaxation (n_0, n_1)\). We have proven \(r_0 - r_1 \geq s_0 - s_1\). Next, we prove \(r_0 \geq s_0\) by means of case distinction. 
   
   Let \(n_0 \geq m_1\).
   \begin{align}
      s_0   &= m_0 - m_1 + \max(m_1, n_0) \\
            &= m_0 - m_1 + n_0 \\
            &\leq p_0 - p_1 + n_0 \label{eqn:3.9}\\
            &\leq p_0 - p_1 + q_0 \label{eqn:3.10}\\
            &\leq p_0 - p_1 + \max(p_1, q_0) \\
            &= r_0
   \end{align}
   
   Where \cref{eqn:3.9} and \cref{eqn:3.10} are justified by \((p_0, p_1) \relaxation (m_0, m_1)\) and \((q_0, q_1) \relaxation (n_0, n_1)\), respectively. Next, we prove that \(r_0 - r_1 \geq s_0 - s_1\).

   Let \(n_0 < m_1\). 
   \begin{align}
      s_0   &= m_0 - m_1 + \max(m_1, n_0) \\
            &= m_0 - m_1 + m_1 \\
            &= m_0 \\
            &\leq p_0 \label{eqn:3.22}\\
            &\leq p_0 - p_1 + \max(p_1, q_0) = r_0 \label{eqn:3.23}
   \end{align}

   Where \cref{eqn:3.22} uses \((p_0, p_1) \relaxation (m_0, m_1)\), and \cref{eqn:3.23} uses the fact that \(\max(p_1, q_0) - p_1 \geq 0\) for any \(p_1, q_0\).

   This concludes our proof of soundness.
\end{proof}
