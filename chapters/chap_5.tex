\chapter{Adding Variables}

Before continuing, let us examine the limitations of the \nameref{def:prog-lang-4} from the previous chapter. We are not able to declare variables, control flow, or constants. All of those features will be implemented in this chapter. In order to declare and use variables in our programming language, we need to provide a definition of an environment, which comprises the concrete values of variables. As we will see, introducing variables will demand cautious reformulation of most evaluation rules to account for environments.

Similarly to environments, we will introduce contexts for the type rules. While an environment maps variables to concrete values, a context will map variables to types. This will also require carefully embellishing previous type rules, to provide rules that are rigorous with respect to contexts. We define the syntax of the augmented programming language below:


\begin{definition}[Programming Language]
   \label{def:prog-lang-5}

\begin{align*}
   e := ~~~ & \letexp{x}{e_1}{e_2}		& \text{(let)}\\
            & \tick{k}				& \text{(tick)}\\
	    & x					& \text{(var)}\\
	    & \true ~~| ~~\false		& \text{(bool constructors)}\\
	    & q					& \text{(const rational)}
\end{align*}

\end{definition}



