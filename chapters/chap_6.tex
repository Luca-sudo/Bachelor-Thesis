\chapter{VarTick Language}

Before continuing, let us examine the limitations of the \nameref{def:prog-lang-4} from the previous chapter. We are not able to declare variables, control flow, or constants. All of those features will be implemented in this chapter. In order to declare and use variables in our programming language, we need to provide a definition of an environment, which comprises the concrete values of variables. As we will see, introducing variables will demand cautious reformulation of some evaluation rules to account for environments.

Similarly to environments, we will introduce contexts for the type rules. While an environment maps variables to concrete values, a context will map variables to types. This will also require carefully embellishing previous type rules, to provide rules that are rigorous with respect to contexts. We define the syntax of the augmented programming language below.

Before we can define the syntax, we have to define the set of variable names (\(\Var\)). For our language, valid variable names are those that comprise only letters, without whitespace and numbers. 

\begin{definition}[VarTick syntax]
   \label{def:prog-lang-5}

\begin{align*}
   e := ~~~ & \letexp{x}{e_1}{e_2}		& \text{(let)}\\
            & \tick{k}				& \text{(tick)}\\
	    & x					& \text{(var)}\\
	    & \true ~~| ~~\false		& \text{(bool)}\\
	    & k					& \text{(integers)}\\
\end{align*}
\end{definition}

The newly introduced constructors for integers and booleans are reflected in the type system for the language:

\begin{definition}[VarTick types]\label{def:type-system-5}
   \[
      \type := \unit~|~\text{\bool}~|~\text{\typeint}
   \]
\end{definition}

In order to give a sound definition of the set of values (\Vals), we define the set of values for each type respectively. 

\begin{align*}
   \Vals_{\unit} & = \{\bullet\} \\
   \Vals_{\bool} & = \{\true, \false\} \\
   \Vals_{\typeint} & = \mathbb{Z}
\end{align*}

The set of \emph{all} possible values is simply the disjoint union of the set of values of each type. Note that \(\sqcup\) denotes the disjoint union of sets.

\begin{definition}
   \(\Vals = \Vals_{\unit} \sqcup \Vals_{\bool} \sqcup \Vals_{\typeint}\)
\end{definition}

We also want to express that a variable has a specific type, which is the same as claiming that the variable is in the set \(\Vals_{A}\) for the appropriate type \(A\). Thus, we write \(v : A\) to denote \(v \in \Vals_{A}\).


\section{Evaluation Semantics}

Expressions now feature variables, necessitating the introduction of \emph{environments} to resolve variables to their associated value. In \cref{sec:environments} we define environments, afterward we introduce evaluation rules in \cref{sec:evaluation-rules-var-tick}.

\subsection{Environments}\label{sec:environments}

Environments map variable names to values. Here, \(\Var\) is the set of variable names, and \(\Vals\) is the set of values. 

\begin{definition}[Environment]\label{def:environment}
   An environment \(E: \Var \partialto \text{\Vals}\) is a partial function.
\end{definition}

For an environment \(E\), we may be interested in the set of variable names that are defined. This is called the \emph{domain} of \(E\). As some evaluation rules required the existence of certain variables, we use the domain to check variable names for membership. For this, we write \(x \in \domain{E}\) for \(x \in \Var\).

\begin{definition}[Domain of environments]
   Let \(E\) be an environment. The \emph{domain} \(\domain{E}\) is the set of all variable names that are defined in \(E\).
\end{definition}

In most programming settings, we start with an empty environment and incrementally populate it with variables as our program evaluates. This constitutes the need for extending environments - we want to add variables to the environment upon introducing them.

\begin{definition}\label{def:environment-augment}	
   Let \(E\) be an environment, \(x \in \Var\) and \(v \in \Vals\) be the value assigned to \(x\). We write \(E[x \mapsto v]\) for the environment \(E\), where \(x\) is \emph{additionally} mapped onto \(v\).
\end{definition}

Having all of these definitions in place, we now update the definition of an \emph{evaluation judgement}. Previously, expression were evaluated as is, as no environment influenced the execution of the expression. Now, we consider evaluating expression against an environment \(E\).

\begin{definition}[Evaluation Judgement with Environments]\label{def:eval-judgement-environments}
   Let \(E\) be an \nameref{def:environment}, \(e\) an expression of our programming language (\cref{def:prog-lang-5}), \(v\) the value obtained from evaluating \(e\), and let \((p_0, p_1) \in \resource\). We then write the following for the evaluation judgement with respect to an environment:

   \[
      \evals{E}{e}{v}{p_0}{p_1}
   \]
	
\end{definition}

\begin{remark}
   Note that the evaluation of an expression \(e\) depends on the environment, due to the ambiguity of variable names. Thus, both the return value \(v\) and the resource demand \((p_0, p_1)\) can differ for discrete environments. 
\end{remark}

\subsection{Evaluation Rules}\label{sec:evaluation-rules-var-tick}

Having all the necessary definitions at hand, let us now define evaluation rules for our programming language. We start with the rule (E:var), which provides an explicit handle for resolving variable names to their values. In order for the rule to be consistent, we must enforce that the variable we want to resolve is defined in the current environment. 

\[
   \inference[(E:var)]
   {\contains{E}{x}}
   {\evals{E}{x}{E(x)}{0}{0}}
\]

Next, we define one rule (E:bool) that unifies the evaluation of both boolean constants. Similarly, we define the evaluation of constant integers using the rule (E:Int). In contrast to the previous rule, we do not need to check the environment for existence of any variables, as both booleans and integers are evaluated in a direct manner.

\[
   \inference[(E:bool)]{b \in \{\true, \false\}}{\evals{E}{b}{b}{0}{0}}%
   \quad\quad\quad
   \inference[(E:int)]{n \in \mathbb{Z}}{\evals{E}{n}{n}{0}{0}}
\]

Next, we introduce the evaluation rule for \textbf{tick}. Here, the concrete environment has no impact on the evaluation of a tick expression, it is evaluated \emph{as is}. 

\[
   \inference[(E:tick+)]
   {k \geq 0}
   {\evals{E}{\tick{k}}{\bullet}{k}{0}}
   \qquad
   \inference[(E:tick-)]
   {k < 0}
   {\evals{E}{\tick{k}}{\bullet}{0}{|k|}}
\]

The last rule missing is (E:let). Here, we need to carefully weave the value that is the result of the first expression into the second expression. We do so by augmenting the environment using the variable name and result of the preceding sub-expression.

\[
   \inference[(E:let)]
   {\evals{E}{e_1}{v_1}{p_0}{p_1} \qquad \evals{\envaugment{E}{x}{v_1}}{e_2}{v_2}{q_0}{q_1}}
   {\evals{E}{\letexp{x}{e_1}{e_2}}{v_2}{r_0}{r_1}}%
\]
\begin{align*}
   \text{where }  &(r_0, r_1) = (p_0, p_1) \sequence (q_0, q_1)
\end{align*}


\section{Type Rules}\label{sec:type-rules-5}

To type expressions with variables, knowledge of a variable's type is paramount. This necessitates the definition of \emph{contexts} (\cref{def:context}), along with ergonomic notations. The introduction of contexts requires extending \emph{typing judgements} (\cref{def:type-judgement-context}). With these in place, we define the \emph{inference rules} (\cref{sec:inference-rules-letvar}) that are context-aware.

\subsection{Contexts}

What environments are for evaluation rules, contexts are for inference rules. Contexts allow us to map variables to types, which is necessary since introducing variables. Similar to \nameref{def:environment}s, a context is a partial function; Conversely to environments, a context maps variable names to a type from \cref{def:type-system-5}.

\begin{definition}[Context]\label{def:context}
   We define a context \(\Gamma : \Var \partialto \type\) as a partial function from variable names to types. 
\end{definition}

Most definitions for contexts are similar to the definitions for environments. For example, we define the \emph{domain} of a context to comprise all variables that are defined.


\begin{definition}[Domain of Context]\label{def:context-domain}
   Given a context \(\Gamma\), the domain \(\domain{\Gamma}\) is the set of all variables that are defined in \(\Gamma\).
\end{definition}

To assert that a variable \(x\) is defined in the current context \(\Gamma\), we write \(x \in \domain{\Gamma}\).

While executing a program, multiple variables with the name \(x\) may be created at discrete times. It is important to note that the \emph{most recent} initialization will determine the value of \(\Gamma(x)\) - for let expressions, this can lead to inconsistencies. \emph{Disjoint} contexts circumvent this problem, as two disjoint contexts do not share any variables, eliminating the chance for inconsistencies.

\begin{definition}[Disjoint contexts]\label{def:disjoint-contexts}
   Let \(\Gamma_1, \Gamma_2\) be contexts. 

   \[
      \Gamma_1 \text{ and } \Gamma_2 \text{ are disjoint} :\iff \domain{\Gamma_1} \cap \domain{\Gamma_2} = \emptyset
   \]
\end{definition}

At the start of executing a program the context is empty. During execution, variables are initialized; This requires extending the context to account for the newly initialized variable. Given that \(x : A\) we add \(x\) to \(\Gamma\) by writing \(\Gamma~;~x : A\). However, we require that \(x \notin \domain{\Gamma}\). In a sense we require that \(\Gamma\) and the context containing only \(x\) are \emph{disjoint}.

With contexts defined, we embed them into our new definition of a \emph{typing judgement}.

\begin{definition}[Typing Judgement with contexts]\label{def:type-judgement-context}
   Let \(\Gamma\) be a context, \(e\) an expression of \cref{def:prog-lang-5} with the return type \(A\), and let \((p_0, p_1) \in \resource\). We then denote a typing judgement with respect to the context \(\Gamma\) by writing:
   \[
      \typing{\Gamma}{e}{\ratype{A}{p_0}{p_1}}
   \]
\end{definition}

\begin{remark}
   Similarly to \cref{def:eval-judgement-environments}, the resource-aware type assigned to \(e\) will vary on the context in which \(e\) is evaluated. 
\end{remark}

\subsection{Inference Rules}\label{sec:inference-rules-letvar}

Let us start by introducing the rule (T:var), which allows us to add a variable and its associated type into a context.

\[
   \inference[(T:var)]
   {}
   {\typing{\Gamma; x: A}{x}{A}}
\]

Next, we introduce two rules, one for boolean constants and one for integer constants. The rule (T:bool) assigns a type \bool~to any occurrence of \(\true\) or \(\false\), and similarly, the rule (T:int) assigns a type \typeint~to any occurrence of integer constants.

\[
   \inference[(T:bool)]
   {b \in \{\true, \false\}}
   {\typing{\Gamma}{b}{\text{\bool}}}%
   \qquad
   \inference[(T:int)]
   {n \in \mathbb{Z}}
   {\typing{\Gamma}{n}{\text{\typeint}}}
\]

Defining the type rule for the \(\tick{}\) is straightforward, we add a context \(\Gamma\) to the rule from the previous chapter \cref{def:tr-tick-4}. 

\[
   \inference[(T:tick)]
   {q_0 \geq k + q_1}
   {\typing{\Gamma}{\tick{k}}{\ratype{\unit}{q_0}{q_1}}}
\]

\begin{remark}
   As the sub-expressions of the let expression might have different contexts, we need to account for that as well. To permit a consistent formulation of the type rule, we need to add an assumption: the two contexts need to be \emph{disjoint}. This will allow us to avoid any typing inconsistencies that could arise from mismatched types between the two contexts. 
   For this, a variable \(x\) could be present in both contexts \(\Gamma_1\) and \(\Gamma_2\), with different types assigned to the variable:

   \[
      \typing{\Gamma_1}{x}{A} \qquad \typing{\Gamma_2}{x}{B}
   \]

   This is prevented, by demanding that the contexts are \emph{disjoint}. While this makes the rule less flexible, we avoid cases where the application of (T:let) would yield a possibly ill-typed judgement; If the variable \(x\) is present in both contexts, altering the type of that variable may construe the resulting typing of the expression.
\end{remark}

This leads to the following formulation of the rule (T:Let). Compared to the previous chapter, we now require the contexts to be disjoint, and propagate the type of \(e_1\) to the context \(\Gamma_2\) to type \(e_2\). Previously, no variables were acrued and thus no propagation was needed.

\[
   \inference[(T:let)]
   {\typing{\Gamma_1}{e_1}{\ratype{A}{p_0}{p_1}} \qquad \typing{\Gamma_2;x : A}{e_2}{\ratype{B}{q_0}{q_1}} \qquad \Gamma_1 \cap \Gamma_2 = \emptyset}
   {\typing{\Gamma_1; \Gamma_2}{\letexp{x}{e_1}{e_2}}{\ratype{B}{p_0}{q_1}}}
\]


\section{Soundness}

Having introduced \nameref{def:environment}s and \nameref{def:context}s, it is necessary to formulate the soundness theorem as to account for environments and contexts. This yields the following statement:

\begin{theorem}[Soundness of typing for var-tick language]\label{thm:soundness-5}
   Let \(E\) be an environment, \(\Gamma\) be a context. Further, let \(A\) be a type, \((p_0, p_1), (q_0, q_1) \in \resource\) and an expression \(e\) that evaluates to the value \(v\). The Soundness Theorem states:

   \begin{center}
   If \(\evals{E}{e}{v}{p_0}{p_1}\) and \(\typing{\Gamma}{e}{\ratype{A}{q_0}{q_1}}\), then \(v : \ratype{A}{}{}\) and \((q_0, q_1) \relaxation (p_0, p_1)\).
   \end{center}
\end{theorem}

\begin{proof}
   The proof uses the same structure from the previous chapter (\cref{thm:soundness-let-tick}). We perform structural induction over the type derivation. 

The proof given in \cref{thm:soundness-let-tick} still holds for the rules (T:Tick) and (T:Let).

(T:Bool): We know that \(p_0 = p_1 = 0 = q_0 = q_1\) by definition of the evaluation and typing rule. As a result \(p_0 \geq q_0\) and \(p_0 - p_1 \geq q_0 - q_1\), which is identical to \((p_0, p_1) \relaxation (q_0, q_1)\). It furthermore follows that the type of the resulting value \(b\) is of type \bool, and such, consistent with the typing derived from the type rule.

(T:Int): Identical to the previous proof, we get that \(p_0 = p_1 = 0 = q_0 = q_1\) which shows \((p_0, p_1) \relaxation (q_0, q_1)\), and similarly, the type of \(n\) is \typeint, which is consistent.

\end{proof}





