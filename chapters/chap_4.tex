\chapter{Rudimentary Type System}

In this chapter, we define the type system, along with the associated operational semantics and type rules, for a simple system, that only counts the amount of ticks in a program. The chapter concludes with a soundness proof, showing that operational semantics and type rules line up.
In order to elicit resource bounds for programs, we need two things: (1) A construct in the programming language that allows altering the calculated resource consumption and (2) types that reflect resource consumption. The first problem is solved by introducing \(\text{tick q}\) to the programming language. Where \(q\) is a rational number, indicating the amount of resources that are either consumed of freed. The second problem is solved by embellishing the unit type with two resource annotations, one for the resources required a priori, and another for the resources freed afterward.

In the figure below is the programming language that we use:

\begin{align*}
   e := ~~~ & \text{let } \_ = e_1 \text{ in } e_2   & \text{(let)}\\
            & \text{tick } q                       & \text{(tick)}
\end{align*}


\section{Operational Semantics}
In this section, we define what it means for a program to consume (or free) resources, by defining the operational semantics needed. To this end, let \(H\) be a heap \(e\) be an expression and \(v\) be the value produced by the expression \(e\). We then write \(H \vdash e \rightsquigarrow v, H'\) for the fact that given heap \(H\) the expression \(e\) evaluates to the value \(v\) and yields the new heap \(H'\). In order to capture resource-consumption, we augment this notation like this \(H \vdash e \rightsquigarrow v, H'~| ~(p, p')\), where \(p\) and \(p'\) are integers and \(p\) being the amount of resources needed to start evaluating the expression, and \(p'\) being the resources freed after the evaluation of expression \(e\).

When composing evaluations, we need to define sound rules for the resulting resource consumption. We will build up an ergonomic and sound definition step by step. Firstly, we need to understand how the resource consumption of subsequent expressions behaves. For this, there are two cases: The first operation reimburses enough resources for the second operation to be run, or the second operation consumes more resources than the first operation reimburses. Concretely, for two evaluations \( \vDash e_1 \rightsquigarrow v_1 | (p_0, p_1)\) and \( \vDash e_2 \rightsquigarrow v_2 | (q_0, q_1)\), we define the resulting cost of their composition as follows:
\[
   \begin{cases}
      (p_0 + q_0 - p_1, q_1) & \mbox{if } q_0 \geq p_1 \\
      (p_0, q_1 + p_1 - q_0) & \mbox{if } q_0 <    p_1
   \end{cases}
\]

This definition ensures that we do not "erase" required resources out of a sequence of evaluations. As an example, consider a program comprising two expressions. The first one consuming 2 resources, and the second one freeing up 3 resources. Without the case distinction above, we could "erase" the \(2\) resources that are required a priori. 

The first case is relevant, if the resources required by the second operation are exceeded, which is equivalent to \(q_0 \geq p_1\). In this case, we need to increase the initial resources by the difference between the reimbursement of the first operation and the cost of second operation. For the second case, the resources that are reimbursed suffice to start the second operation. As a result, we do not need to increase the initial cost, but increase the final reimbursement. It becomes apparent, that the difference between the resource reimbursement of the first operation and the resource requirement of the second operation has a key role in determining the resulting cost. We, therefore, give this metric a distinguished name:

\begin{definition}
   Let \((p_0, p_1)\) and \(q_0, q_1)\) be two resource pairs, of subsequent operations, as introduced above. We then define the \textbf{disparity} of these two operations as \(\Delta = \max(p_1, q_0)\).
\end{definition}


\begin{definition}

   Let \((p_0, p_1)\) and \((q_0, q_1)\) be two resource pairs, as introduced above. We then define 
   \((p_0, p_1) \cdot (q_0, q_1) = (p_0 - p_1 + \Delta, q_1 - q_0 + \Delta)\)
   
\end{definition}

Having defined the notion of evaluation, we now introduce rules for the operational semantics provided.
First, we need a rule that allows us to combine two expression into a let expression. We also need to, however, provide resource annotations for the resulting let expression. Therefore, we also need to add numeric constraints, resulting in the following rule:

\[
   \inference[(let)]
   {H_1 \vdash e_1 \rightsquigarrow v~H_1'~|~(p,p')~~,~~ H_1' \vdash e_2 \rightsquigarrow v_2~H_2'~|~(q, q')}
   {H_1 \vdash \<let> ~ \_ = e_1 ~ \<in> ~  e_2~|~(p, p') \cdot (q, q')}
\]

Similarly, we now define a rule for the \<tick> instruction. The expression \<tick> \(q\) consumes precisely \(q\) resources; or frees \(q\) resources, should the amount be negative. As a result, any resource-annotation where a priori and a posteriori resources differ by at least \(q\), is valid. This leads to the following rule:

\[
   \inference[(tick)]
   {q_0 \geq q + q_1}
   {H \vdash \<tick> q \rightsquigarrow~|~(q_0, q_1)}
\]

In the next section, we introduce type rules that are tightly linked to the operational semantics defined now. Afterward, we will prove the soundness theorem, providing a correspondence between operational semantics and type rules.

\section{Type Rules}
The only resource-annotated type that we permit is the unit type, denoted by \(\unit^p_r\), where \(p\) provides a lower bound on the resources needed in order to evaluate an expression of that type, and \(r\) provides a bound on the amount of resources that are freed after evaluating an expression of that type. 

Similarly to the previous section on operational semantics, we define two rules, one for \<let> expressions and one for \<tick>. Before we can introduce the rule for \<let>, however, we need to define how resource-annotations of types propagate. For this, consider the following example:

\[
   \inference[]
   {\Gamma_1 \vdash e_1 : \unit^p_r~~,~~ \Gamma_2 \vdash e_2 : \unit^s_t}
   {\Gamma_1 \Gamma_2 \vdash \<let> ~ \_ = e_1 ~ \<in> ~ e_2 :~?}
\]

It is not immediately clear, what type the resulting \<let> expression should have. We, therefore, define an operation on types that is closely linked to the definition ... in the previous section.
\todo{link to definition}

\begin{definition}
   Let \(\unit^{p_0}_{p_1}\) and \(\unit^{q_0}_{q_1}\) be resource-annotated types. \\
   We define \(\unit^{p_0}_{p_1} \ltimes \unit^{q_0}_{q_1} = \unit^{p_0 - p_1 + \Delta}_{q_1 - q_0 + \Delta}\)
\end{definition}

\todo{Add that this forms a monoid that can be extended to accomodate more than just unit as well?}

The two case distinctions of the definition correspond to two, distinct, cases. In the first case, the cost of the second operation is higher than the reimbursement of the first operation - we therefore need to "pay" for the difference. In the second case, the reimbursement of the first operation completely accounts for the cost of the second operation. As a result, we do not have to provide additional resources, but instead additional resources are freed.

Using the newly defined operation on resource-annotated types, we get the following rules:

\[
   \inference[(T:let)]
   {\Gamma_1 \vdash e_1 : \unit^{p_0}_{p_1}~~,~~ \Gamma_2 \vdash e_2 : \unit^{q_0}_{q_1}}
   {\Gamma_1 \Gamma_2 \vdash \<let> ~ \_ = e_1 ~ \<in> ~ e_2 :~\unit^{p_0}_{p_1} \ltimes \unit^{q_0}_{q_1}}
\]

\[
   \inference[(T:tick)]
   {q_0 \geq q + q_1}
   {\Gamma \vdash \<tick> ~ q : \unit^{q_0}_{q_1}}
\]

Let us now perform an examplary type derivation for a simple program, using the type rules we just introduced, to get a better understanding of how the definitions orchestrate together. For this, consider the following program:

\begin{align*}
  & \<let> \_ = \<tick>~3~\<in> \\
  & \<let> \_ = \<tick>~ -2 ~\<in> \\
  & \<tick>~5 \\
\end{align*}

Intuitively, we have three distinct steps, the first one consuming 3 resources, the second one freeing up 2 resources and the last one consuming 5 resources. Let us now derive a typing for the program.

\[
   \inference
   {
      \inference
      {q_0 \geq 3 + q_1}
      {\<tick> ~ 3 : \unit^{q_0}_{q_1}}
       &
      \inference
      {
         \inference
         {t_0 \geq -2 + t_1}
         {\<tick> -2 : \unit^{t_0}_{t_1}}
          &
         \inference
         {s_0 \geq 5 + s_1}
         {\<tick>~5 : \unit^{s_0}_{s_1}}
      }
      {\<let> ~ \_ = \<tick> -2~\<in>~\<let>~\_ = \<tick>~5 :~\unit^{t_0}_{t_1} \ltimes \unit^{s_0}_{s_1}}
   }
   {\<let> \_ = \<tick>~3~\<in> ~ \<let> \_ = \<tick>~-2~\<in>~\<let> \_ = \<tick>~5 :\unit^{q_0}_{q_1} \ltimes \unit^{t_0}_{t_1} \ltimes \unit^{s_0}_{s_1}}
\]

We are now left with the task of resolving the collected, numeric constraints. A possible solution to the system of constraints the following: \(q_1 = t_0 = s_1 = 0\) , \(q_0 = 3\), \(t_1 = 2\) and \(s_0 = 5\). This completes our type derivation, as inserting the calculated resource-annotations yields our desired type. The resulting type is: \(\unit^{3}_{0} \ltimes \unit^{0}_{2} \ltimes \unit^{5}_{0}\).
Applying the definition of the \(\ltimes\) operator, allows us to shrink the type, which we do stepwise for sake of illustration:

\begin{align*}
   \unit^{3}_{0} \ltimes \unit^{0}_{2} \ltimes \unit^{5}_{0} &\stackrel{(1)}{=} \unit^{3 + 0 - 0}_2 \ltimes \unit^5_0\\
                                                             &\stackrel{(2)}{=} \unit^{3 + 5 - 2}_2 = \unit^6_2
\end{align*}

Where we perform the first step, because \( 0 \geq 0\), which are \(q_1\) and \(r_0\) in the definition respectively. And we perform the second steps, as \(5 \geq 2\), which is the other case distinction in our definition.

This indeed furnishes a tight bound on the resources required. Especially for proving the soundness theorem, which is the goal of the next section, introducing a third rule makes the proof easier. We introduce a rule that allows us to \emph{relax} the resource-bounds of a type. For illustration, consider the type \(\unit^4_0\). \emph{Relaxing} this type allows u to also derive a type like \(\unit^6_2\), where we simply incremented both the a priori and a posteriori resource by two units, respectively. When defining the type rule, we need to be cautious, as the relaxed type should not have a diminished resource consumption. As such, we \emph{do not} want to relax the type \(\unit^4_0\) into the type \(\unit^4_2\), as this would reduce the resource consumption.

\[
   \inference[(T:relax)]
   {\Gamma \vdash () : \unit^{p_0}_{p_1}
      & 
   q_0 \geq p_0
      &
   q_0 - p_0 \geq q_1 - p_1}
   {\Gamma \vdash () : \unit^{q_0}_{q_1}}
\]


\section{Soundness Theorem}
Ultimately, we want our type rules to be consistent with the operational semantics defined previously. This will allow us to reason about the type of a program, while ensuring that the resource-annotations of the resulting type line up with the concrete resource consumption of the expression.
The proof of soundness comprises multiple parts. To this end, we need to introduce the notion of a \emph{well-formed} environment, as this provides a necessary congruence between a typing context \(\Gamma\) and an environment \(V\), that is needed, to prove soundness.

We say that an environment \(V\) is \emph{well-formed} with respect to a typing context \(\Gamma\), if the types that are assigned to variables in \(\Gamma\) are consistent with the types that are assigned to variables in the environment \(V\). The importance of this constraint can best be understood by a counterexample, which uses types that are used in programming languages such as C++, Rust, etc. Consider a context \(\Gamma\) with only one variable, \( x : \text{int}\), and an environment \(V\) for which the variable \(x\) has the type float. It is apparent that the types int and float are not the same. We avoid this by constraining the environment to be well-formed with respect to a context.

Having this definition in place, allows us to sketch the structure of a soundness proof:

If \(\Gamma \vdash e : \unit^{p_0}_{p_1}\) and \(V \vDash e \rightsquigarrow  () | (q_0, q_1) \), where the environment \(V\) is well-formed with respect to \(\Gamma\), then the following hold true:
The result \(()\) of the expression \(e\) has a type that is consistent with the one assigned to \(e\). The a priori resource annotation of the type is not exceeded in the concrete evaluation of the expression \(e\). The resource consumption assigned to the type is not exceeded by the evaluation, either. We can state the same more precisely, like so: 

\begin{itemize}
   \item \(() : \unit^{p_0}_{p_1}\) 
   \item \(p_0 \geq q_0\)     
   \item \(p_0 - p_1 \geq q_0 - q_1\)
\end{itemize}

The first bullet point capturing that the result of evaluating the expression \(e\) has the type that was constructed using the type rules. The second bullet point stating that the \emph{a priori} resource bound that was constructed using the type rules is an upper bound on the concrete resources required a priori. And the third point stating that the resource consumption of the constructed type also provides an upper bound on the concrete resource consumption.

In the first part, I prove that the inequality \(p \geq q\) holds. We use structural induction, which concretely means, we assume that the expression \(e\) has the type \(\Gamma \vdash e : \unit^{p_0}_{p_1}\) and that the expression evaluates, like so \(\vDash e \rightsquigarrow () | (q_0, q_1)\). We now need to prove that, for every type rule that could be the last rule applied in a type derivation, the three bullets points above hold. 

(tick): By assumption, we get that \(\Gamma \vdash \<tick> q : \unit^{p_0}_{p_1}\), where \(p_0 \geq q + p_1\) and \(\vDash \<tick> q \rightsquigarrow () | (q_0, q_1)\), with \(q_0 \geq q + q_1\). 
We make a case destinction. Let \(q \geq 0\). The constraint for the evaluation is then most generally solved with \(q_0 = q\) and \(q_1 = 0\). 

(let): By assumption, we get that \(\Gamma \vdash \<let> \_ = e_1 \<in> e_2 : \unit^{p_0}_{p_1} \ltimes \unit^{q_0}_{q_1}\), and similarly \(\vDash \<let> \_ = e_1 \<in> e_2 \rightsquigarrow () | (r_0, r_1) \cdot (s_0, s_1)\). We do, however, also know by assumption that the following inequalitites hold for the subexpressions \(e_1, e_2\): \(p_0 \geq r_0\) and \(q_0 \geq s_0\), as well as \(p_0 - p_1 \geq r_0 - r_1\) and \(q_0 - q_1 \geq s_0 - s_1\). Our first goal is to prove that \(p_0 - p_1 + \max(q_0, p_1) \geq r_0 - r_1 + \max(s_0, r_1)\). We arrive at this inequality, by first contracting the type in the first assumption, and, by secondly multiplying the resource-annotated tuples in the second assumption, concerning evaluation. 
Showing this inequality reduces to showing that \(\max(q_0, p_1) \geq \max(s_0, r_1)\), as \(p_0 - p_1 \geq r_0 - r_1\) by assumption. It also holds that, by assumption, \(q_0 \geq s_0\). Thus, we only need to show that \(p_1 \geq r_1\). 
















