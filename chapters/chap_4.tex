\chapter{Resources}

Throughout this thesis, we introduce programming language constructs, in order to elicit resource consumption of programs. It is, therefore, important to mention that resources can be an arbitrary cost metric, including but not limited to: memory consumption, runtime, energy consumption. While the details of the respective type of resource are nuanced, the resource consumption generally comprises two parts: Resources consumed to start evaluating, and possibly, resources freed after evaluation.

Tracing the heap-space usage of a program may look like the diagram on the left-hand side. It both consumes and frees resources in intermediary steps. For time consumption, things are different. We cannot reimburse time, but only consume it - as illustrated in the right-hand side example.

\begin{align*}
   \tikzfig{resource_intro} & \quad & \tikzfig{time_illustration}
\end{align*}

We are most interested in the value at the beginning and at the end, which leads us to simplify the resource consumption further. Drawing a straight line between resources at the start and at the end.

\ctikzfig{resource_reduced}

It is important to note that distinct programs may have the same start and end resources, while they differ vastly in their temporal development. In order to reflect this in a resource diagram, we draw a polygon which contains \emph{every} valid development for the specific beginning and end resource amount.

\ctikzfig{polygon_intro}

We use this visual representation to foster intuition throughout this chapter - illustrating examples and concepts.

This overarching structure of resource demands leads us to a refined version of \href{def:resource-consumption}{resource consumption}. Furthermore, we will define a rich structure on the resource demand of programs; Most notably, we introduce sequencing of resource demands, which gives rise to a monoidal structure, and relaxation, which defines a \emph{partial order} - allowing us to compare resource demands, in a way that is consistent with our intuition of cost.

\section{Resource Demand}

We start by defining resources mathematically, which will lead to defining resource demand of programs as a tuple of resources. Afterward, we introduce multiplication of these tuples and disambiguate reasoning about resource demand, by naming key figures of resource demands.

\begin{definition}[Resources]\label{def:resources}
   We identify resources with the natural numbers, where the value refers to an amount of resources.
   \[
      \resource = \mathbb{N}_0
   \]
\end{definition}


\begin{definition}[Resource demand]\label{def:resource-pair}
   We denote by \(\demand\) the set of resource demands, which is a pair of resources.
   \[
      \demand = \resource \times \resource
   \]
\end{definition}

When reasoning about a resource demand \((p_0, p_1) \in \demand\), \(p_0\) are the \emph{initial} resources, and \(p_1\) are the \emph{residual} resources. 

\begin{example}
   \((4, 2) \in \demand\)

Representing this resource demand using resource diagrams yields:

\ctikzfig{ex25}

\end{example}

For this example, one might wonder if we can simplify \((4, 2)\) to \((2, 0)\), by simply shifting the polygon downward. This is not possible, as a program with demand \((4, 2)\) may not have sufficient resources if it is initiated with only \(2\) initial resources. 

Consider the diagrams below. On the left-hand side, the development of the program is contained in the associated polygon. However, on the right-hand side, this is not the case. The amount of available resources is even negative at some point - this is not valid. 

\begin{align*}\label{fig:resource-shift-down}
   \tikzfig{shift_down_1} & \quad \tikzfig{shift_down_2}
\end{align*}

On the other hand, one may ask if a program in \((2, 0)\) can also be represented by \((4, 2)\). The answer is yes! Further clarification as to why this is the case is subject of \cref{sec:comp-pairs}.

The notion of \emph{consumption} is nuanced. On the one hand, the example above may use all four resources at some point, but frees two resources after its execution. For our use, we view the demand of programs as a black box, only concerned with initial and residual resources. As a result, the \emph{consumption} of a resource demand is precisely the difference between initial and residual resources. 

\begin{definition}[Resource Consumption]\label{def:resource-consumption}
   Given \((p_0, p_1) \in \demand\), we define the \emph{resource consumption} as the difference between initial and residual resources, \(p_0 - p_1\).
\end{definition}

While a resource demand is a pair of \emph{non-negative} integers, its resource consumption can be negative - this corresponds to a program that \emph{frees} resources instead of consuming them. We can think of the resource consumption as the directed (!) height of the associated resource diagram. Consider the following example.

\begin{example}
   For \((2, 4) \in \resource\), the resource consumption equals \(2 - 4 = -2\). An associated programs \emph{frees} 2 resources!

\ctikzfig{resource_consumption}

\end{example}

As most programs are made up of multiple subsequent evaluations, we need a mechanism for composing evaluations; More specifically, we need to define the resource cost of subsequent evaluations. 
For this, there are two cases: The first operation has enough residual resources for the second operation to be executed, or the second operation requires more initial resources than residual resources of the first operation provides. In the latter case we need to increase the initial resources of the composed evaluation.

This yields two cases for the resulting resource demand:
\[
   \begin{cases}
      (p_0 + q_0 - p_1,  q_1) & \mbox{if } q_0 \geq p_1 \\
      (p_0,q_1 + p_1 - q_0) & \mbox{if } q_0 <    p_1 
   \end{cases}
\]

For the first case, we lift the first resource demand so its residual resources match the initial resources of subsequent resource demand - allowing us to compose both resource demands.

\begin{align*}
   \tikzfig{sequencing_11} & \implies & \tikzfig{sequencing_12}
\end{align*}

Intuitively, one may want to lower the second resource demand to match. This, however, is not a valid composition. Consider the same resource demands, but with an example progression outlined in the second resource demand. If we were to lower the second resource demand, we would have negative resources at some point in our evaluation, similar to \cref{fig:resource-shift-down} - this would not be sound.

\begin{align*}
   \tikzfig{seq_neg_11} & \implies & \tikzfig{seq_neg_12} 
\end{align*}   

For the second case, we lift the second resource demand to match. We cannot lower the first resource demand, due to reasons congruent with the example above. Fundamentally, composing resource demands forces us to lift either resource demand to match, as lowering any resource demand may reduce the set of programs represented by the resulting resource demand.

\begin{align*}
   \tikzfig{sequencing_21} & \implies & \tikzfig{sequencing_22}
\end{align*}

We can, however, simplify the multiplication of resource demands by merging the two cases. Consider that \(p_0 - p_1 + \max(p_1, q_0)\) precisely matches the initial resources for both cases. Similarly, \(q_1 - q_0 + \max(p_1, q_0)\) precisely matches the residual resources for both cases. 
This yields the following definition:

\begin{definition}[Composition of resource demands]
   \label{def:multiplying-pairs}
   Let \((p_0, p_1)\) and \((q_0, q_1)\) be two \nameref{def:resource-pair}s, as introduced above. We then define:
   \[(p_0, p_1) \sequence (q_0, q_1) = (p_0 - p_1 + \max(p_1, q_0), q_1 - q_0 + \max(p_1, q_0))\]
\end{definition}
\todo{Elaborate/remark: This isnt linear, how can it be solve by LP?}

As the value \(\max(p_1, q_0)\) is key for composing resource demands, we call it the \emph{disparity} of two resource demands.
\todo{Better name}

\begin{example}
   For \((4, 2), (5,0) \in \resource\), we have \((4, 2) \sequence (5, 0) = (7,0)\).

\begin{align*}
   \tikzfig{ex27I} & \implies & \tikzfig{ex27II}
\end{align*}

\end{example}
 
Note that compopsition of resource demands is \emph{not} commutative. As a counterexample, consider \((2, 0) \sequence (0, 2)\) and \((0, 2) \sequence (2, 0)\). The first evaluates to \((2, 2)\), whereas the second evaluates to \((0, 0)\).  

\begin{lemma}
   Resource demands, with unit \((0, 0)\) form a \emph{monoid} under sequencing.
\end{lemma}

\begin{proof}
   \todo{Proof}
\end{proof}


\section{Comparing Resource Pairs}\label{sec:comp-pairs}

Whenever we define a programming language, we provide evaluation semantics and type rules, which we ultimately want to link by proving their \emph{soundness}. Soundness of a programming language with respect to evaluation semantics and type rules comprises two parts. First, we need to prove that the type we assign to an expression is consistent - evaluating the expression will produce a value of that type. Secondly, we need to show that the inferred resource demand is always \emph{at least} the concrete resource demand of evaluating an expression. In subsequent chapters, the main challenge of proving soundness will be in proving consistency of resource demands. 

In order to do this, we need to be able to compare resource demands. This ultimately motivates the definition of \emph{relaxation}, which induces a partial order on the set of resource demands, streamlining future soundness proofs. Let us, therefore, begin by motivating how to compare resource demands in a way that aligns with our goal of AARA.

Intuitively, if a resource demand \(a\) is a \emph{relaxation} of another resource demand \(b\), we expect every program that is executable with respect to \(b\) to also be executable with respect to \(a\). Therefore, \(a\) may have more \emph{initial resources} than \(b\). This makes sense, as a program would then have more resource available throughout its execution. 

Furthermore, we want to compare the \emph{consumption} of both resource demands. In this sense, if \(a\) is a relaxation of \(b\) then \(a\) should consume \emph{at least} as many resources as \(b\). 

\begin{definition}[Relaxation]\label{def:relaxation}
   Let \((p_0, p_1), (q_0, q_1) \in \demand\). We define the relation \relaxation ~as follows:

   \[
      (p_0, p_1) \succcurlyeq (q_0, q_1) :\iff p_0 \geq q_0 \wedge p_0 - p_1 \geq q_0 - q_1
   \]
\end{definition}

We then say that \((p_0, p_1)\) is \emph{a relaxation of} \((q_0, q_1)\). 

Another way to compare resource demands also comes to mind: Instead of the resource consumption to be at least as large, we could enforce that \(p_1 \leq q_1\) - the relaxed resource demand needs \emph{at most} as many residual resources. This would provide a less precise comparison of cost than our definition of relaxation.

To illustrate why, consider the two resource demands \((4, 2), (3, 1) \in \demand\). Any program that can be executed under \((3, 1)\) should also be executable under \((4, 2)\); Intuitively, we provide the program one additional resource. Indeed, \((4, 2) \relaxation (3, 1)\). For the second possible definition of relaxation, we see that \(2 \centernot{\leq} 1\). This highlights the key difference between those definitions - relaxation allows lifting a resource demand, which the second notion prohibits.

Equipped with our definition of relaxation, we illustrate relaxation using resource diagrams. The relaxed resource demand has to start at least the same height \emph{and} be at least as long as the resource demand compared to. 

\begin{example}
   \((4, 2) \relaxation (3, 2)\)

Here both constraints are satisfied, \(4 \geq 3\) and \(4 - 2 \geq 3 - 2\). We see: (1) The first resource demand has a higher starting point, indicating higher initial resources. (2) The first resource demand is vertically longer, which indicates larger resource consumption.

\ctikzfig{ex210}

\end{example}

\begin{example}
   \((4,2) \notrelaxation (5, 4)\)

As \(4 \centernot{\geq} 5\) we have less initial resources. Thus, \((4,2)\) \emph{cannot} be a relaxation. In the diagram below we see that the first resource demand begins at a lower point.

\ctikzfig{ex211}

\end{example}

\begin{example}
   \((5, 4) \notrelaxation (2, 0)\)

For this example it holds that \(5 - 4 \centernot{\geq} 2 - 0\), which violates a constraint for relaxation - the first resource demand consumes \emph{less} resources. In the diagram below, this correlates to the first resource demand being vertically smaller than the second one.

\ctikzfig{ex212}

\end{example}

Most notably, relaxation defines a partial ordering on the set of resource demands. Let us recall the definition of \emph{partial orders}:

\begin{definition}[Partial Order]
   Let \(M\) be a set, and \(\leq\) be a relation on \(M\). A partial order satisfies the following properties for any \(a, b, c \in M\):
   \begin{enumerate}[label=\Roman*]
      \item Reflexivity: \(a \leq a \) 
      \item Anti-symmetry: \(a \leq b\) and \(b \leq a\), then \(a = b\) 
      \item Transitivity: \(a \leq b\) and \(b \leq c\), then \(a \leq c\)
   \end{enumerate}
\end{definition}

Before we can show that resource demands form a partial order under relaxation, we need to define equality for resource demands. Intuitively two resource demands are equal, if the respective initial cost is equal and the resource consumption is equal. 

\begin{lemma}
   The relaxation relation \(\relaxation\) defines a partial ordering on the set of resource demands.
\end{lemma}

\begin{proof}
   In the following, let \(p, q, r \in \demand\).

   \emph{Reflexivity}: It trivially holds that \(p_0 \geq p_0\) and \(p_0 - p_1 \geq p_0 - p_1\), which shows that \(p \relaxation p\).

   \emph{Transitivity}: Let \(p \relaxation q\), \(q \relaxation r\). We want to show that \(p \relaxation r\). Below we list all assumptions that follow, where the left-hand side follows from \(p \relaxation q\) and the right-hand side follows from \(q \relaxation r\).

   \begin{align}
      p_0         &\geq q_0               &  q_0      & \geq r_0 \label{eqn:relaxation-po-1}\\
      p_0 - p_1   &\geq q_0 - q_1         & q_0 - q_1 & \geq r_0 - r_1 \label{eqn:relaxation-po-2}
   \end{align}      

   Using both inequalities in \cref{eqn:relaxation-po-1} and transitivity of \(\geq\) for natural numbers, it follows that \(p_0 \geq r_0\). Similarly, we combine \cref{eqn:relaxation-po-2} and the transitivity of \(\geq\) for the natural numbers to get \(p_0 - p_1 \geq r_0 - r_1\). Both inequalities show that \(p \relaxation r\).

   \emph{Anti-symmetry}: Let \(p \relaxation q\) and \(q \relaxation p\). Again, we list all inequalities that follow from \(p \relaxation q \) and \(q \relaxation p\), respectively.

   \begin{align}
      p_0         &\geq q_0        & q_0        &\geq p_0 \label{eqn:relaxation-po-3}\\
      p_0 - p_1   &\geq q_0 - q_1  & q_0 - q_1  &\geq p_0 - p_1 \label{eqn:relaxation-po-4}
   \end{align}

   Using \cref{eqn:relaxation-po-3} we get \(p_0 = q_0\) and using \cref{eqn:relaxation-po-4} we get \(p_0 - p_1 = q_0 - q_1\). Thus, it follows that \(p = q\).

\end{proof}

It is important to note that, for any two resource pairs, one \emph{need not} be a relaxation of the other. Therefore, relaxation does not define a total order - as the following example shows.

\begin{example}
   Let \((4,2), (5,4) \in \demand\). Then, \((4,2) \notrelaxation (5,4)\) \emph{and} \((5,4) \notrelaxation (4,2)\).
\end{example}

In later chapters (\cref{chap:list-tick}), the introduction of potentials increases the complexity of the proof of soundness (\cref{thm:soundness-7}). To streamline the proof, we introduce a lemma for easier manipulation of relaxations of resource demands.

For an arbitrary resource demand \((p_0, p_1) \in \demand\) and \(n, m \in \resource\), one may ask if \((p_0 + n, p_1 + m) \relaxation (p_0, p_1)\). As it turns out, this is true if and only if \(n - m \geq 0\). This agrees with our intuition: Enforcing that \(n - m \geq 0\) ensures that the resource consumption does not decrease - keeping relaxation constraints sound.

\begin{lemma}\label{lemma:pos-relaxation}
   Let \((p_0, p_1) \in \demand\) and \(n, m \in \resource\). Then,

   \[
      n - m \geq 0 \iff (p_0 + n, p_1 + m) \relaxation (p_0, p_1)
   \]
\end{lemma}

\begin{proof}
   We start by proving \(n - m \geq 0 \implies (p_0 + n, p_1 + m) \relaxation (p_0, p_1)\).

   For the initial resources, \(p_0 + n \geq p_0\) holds, as \(n \in \resource\). For the resource consumption, we get:

   \begin{align*}
      p_0 - n - (p_1 + m)  & = p_0 - p_1 + (n - m) \\
                           & \geq p_0 - p_1
   \end{align*}

   Where we use the assumption \(n - m \geq 0\) from line one to two. 


   We continue with \((p_0 + n, p_1 + m) \relaxation (p_0, p_1) \implies n - m \geq 0\). By assumption,  we get that \(p_0 + n - (p_1 + m) \geq p_0 - p_1\). Thus, \(p_0 - p_1 + (n - m) \geq p_0 - p_1\), which shows that \(n - m \geq 0\). 
\end{proof}

