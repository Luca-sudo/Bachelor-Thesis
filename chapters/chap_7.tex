\chapter{ListTick Language}\label{chap:list-tick}

In this chapter, we introduce the concept of lists as a fundamental data type. This introduction leads to the development of several new definitions, with the most significant among them being potentials. In our previous approach, resource annotations were attached to a type, resulting in a resource cost that followed a "What-you-see-is-what-you-get" principle. However, when dealing with lists, a different approach becomes necessary. Instead of a constant cost for any input list, we assign a specific cost to each individual element within the list. This adaptation allows us to manage resource consumption while accommodating the variable length of lists. The concept of potentials plays a pivotal role in making this adjustment feasible. Beyond our exploration of lists, we delve into function application, which inherently introduces recursion to our programming language. 

\begin{definition}[ListTick syntax]
   \label{def:prog-lang-6}

\begin{align*}
   e := ~~~ & \letexp{x}{e_1}{e_2}		& \text{(let)}\\
            & \tick{k}				& \text{(tick)}\\
	    & x					& \text{(var)}\\
	    & \true ~~| ~~\false		& \text{(bool)}\\
	    & k					& \text{(integer)}\\
            & \cons{head}{tail} ~~| ~~ \listnil & \text{(list constructor)}\\
            & \listmatch{l}{e_1}{e_2}           & \text{(match list)}\\
            & \letFun{f}{x}{e_f}{e_2}           & \text{(function declaration)} \\
            & f~x                               & \text{(function application)}\\
            & \text{where } x \in \Var,~k \in \mathbb{Z},~ head \in A,~ tail \in \ralist{}{A}
\end{align*}
\end{definition}

Because lists can contain \emph{any type} of data, assuming it is homogenous, they are an example of a \emph{polymorphic type}. We can think of polymorphic types as functions from types to types, where the input is called a \emph{type variable} and defines the concrete type of list. 

\begin{definition}[ListTick types]\label{def:type-system-6}
   \[
      A, B := \unit~|~\text{\bool}~|~\text{\typeint}~|~\ralist{}{A}~|~\rafunc{A}{B}{}{}
   \]
\end{definition}

In order to define the set of values for list types, we start by defining the set of values for lists of length \(n\), which we denote by \(\listoflength{n}{A}\). 

\begin{definition}[Lists of Length n]\label{def:lists-of-length-n}
   \(\listoflength{n}{A}\) is the set of all lists of type \(A\) with a length of \(n\).
\end{definition}

The set of values for \(\ralist{}{A}\) comprises lists of every possible length \(n\). Thus, it is the union of \(\listoflength{n}{A}\) for every \(n \in \mathbb{N}\). 

\begin{definition}[Vals for List(A)]
   \begin{displaymath}
      \Vals_{\ralist{}{A}} = \bigcup_{n \in \mathbb{N}} \listoflength{n}{A}
   \end{displaymath}
\end{definition}


We want to infer resource-bounds for programs. In order for the inference to be computationally tractable, we constrain the use of certain language features. Most notably, we enforce that functions on lists only recur on the tail of the list. As a result, recursive function calls \emph{will terminate}, eventually.

While this reduces the set of programs that we permit, most problems on lists naturally have a form that matches this constraint. For example, folding operations such as sum (\cref{fig:code-sum}), or functorial operations such as map (\cref{fig:code-map}), can be defined in this way. 


\section{Evaluation Semantics}

We introduce evaluation rules, which are nuanced when evaluating lists, as the resource demand can be related to the length of the list. This necessitates the introduction of resource demands as \emph{recurrence relations}. We start by introducing recurrence relations and motivate the connection to resource demands. 

\subsection{Recurrence Relations As Resource Demands}\label{sec:recurrence-relation}

Since we introduced lists and recursion into our programming language, this has a profound impact on the relation between input and resource demands. Previously the values of any type in our type system had no \emph{structural implication} for resource demands. As lists can inhabit any (finite) length, we need to find a general approach to relating lists with resource demands. 

Let us build an intuition for the interplay of resource demands and lengths of lists, by informally working through an example function. Consider the function \(add1\) in \cref{fig:code-add1}, which increments every entry in the list.

Our goal is to represent the resource demand \(Q\) of a list \(l\) as a function of the tail of the list. We want to derive an equation of the form \(Q(\cons{x}{xs}) = \alpha + Q(xs)\), assigning a cost to an element \(x\) of the list. To ensure that this equation produces a well-defined result, we need to derive a cost \(Q(\listnil)\) as well.

Given a non-empty list \(l\), we execute the body of the second match condition. There, we tick and increment the head of the list. Afterward we call the function recursively on the tail. Thus, we have paid \(2\) resources for the head of the list, plus the cost of calling the function on the tail, which is not yet determined. This yields an equation of the form 

\[
   Q(\cons{x}{xs}) = 2 + Q(xs)
\]


To derive a complete recurrence relation, and thus a resource demand for arbitrary, finite lists, we need to derive the cost \(Q(\listnil)\). In our example, we simply return nil with no associated cost. Thus, we have \(Q(\listnil) = 0\). Using both equations we define the cost of a list \(l\):

\[
   Q(l) = \begin{cases*}
      2 + Q(xs)            & if l = x :: xs\\
      0                    & else
   \end{cases*}
\]

Understanding this example, we define recurrence relations. Notably, we constrain recurrence relations to have \emph{order} 1, as we only allow recursion on the tail of a list. 


\subsection{Evaluation Rules}

We start by introducing two rules for the construction of lists, corresponding to the two constructors of our programming language. The construction of empty and non-empty lists has no associated cost. As a result, the cost of operations on lists is purely dictated by the actions performed on the list. 

\[
   \inference[(E:Nil)]
   {}
   {\evals{E}{\listnil}{\bullet}{0}{0}}
   \qquad
   \inference[(E:Cons)]
   {\contains{E}{x, xs}}
   {\evals{E}{\cons{x}{xs}}{(E(x), E(xs))}{0}{0}}
\]

In order to implement functions that work recursively on a list, we need pattern matching. The two cases of the pattern match are defined by the inductive definition of lists: (1) A list is nil, or (2) A list comprises head and tail. We introduce one evaluation rule for each case respectively. The first rule matches an empty list.

\[
   \inference[(E:MatchNil)]
   {E(l) = \bullet \qquad \evals{E}{e_1}{v}{q_0}{q_1}}
   {\evals{E}{\listmatch{l}{e_1}{e_2}}{v}{q_0}{q_1}}
\]

Next, we define an evaluation rule for non-empty lists. Here, we need to extract the resource demand as a recurrence relation, which we construct similar to \cref{sec:recurrence-relation}.

\[
   \inference[(E:MatchCons)]
   {E(l) = (v_x, v_{xs}) \qquad \evals{E[x \mapsto v_x,~xs \mapsto v_{xs}]}{e_2}{v}{p_0}{p_1}} 
   {\evals{E}{\listmatch{l}{e_1}{e_2}}{v}{p_0}{p_1}}
\]

To properly evaluate recursive functions we introduce an evaluation rule for function application. We define \(e_f\) as the body of the function \(f\), and the variable name \(y_f\) as the argument used in the definition of \(f\). 

\[
   \inference[(E:Application)]
   {E(f) = e_f 
   \quad 
   \evals{E[y_f \mapsto E(x)]}{e_f}{v}{q_0}{q_1}}
   {\evals{E}{f~x}{v}{q_0}{q_1}}
\]

In order to apply a function, it needs to be defined. The rule (E:LetFun) reads the function name, argument name and the body of the function. It then augments the environment, \emph{without executing the function body}. In later expressions the function identifier is available and may be used, incurring the cost for executing the function body then.

\[
   \inference[(E:LetFun)]
   {\evals{\envaugment{E}{f}{e_f}}{e_{\text{next}}}{v}{p_0}{p_1}}
   {\evals{E}{\letFun{f}{x}{e_f}{e_{\text{next}}}}{v}{p_0}{p_1}}
\]

Note that (1) Function declaration itself has no cost assigned to it (2) We use the keyword \(\textbf{fin}\) to denote the end of the function body. This serves a similar purpose to the keyword \(\textbf{in}\) from let-expressions, as subsequent expression can be written afterward. A key reason for introducing \(\textbf{fin}\) like this stems form the need to distinguish sequential expressions inside the function body \(e_f\) and the end of function declaration. If both used the keyword \(\textbf{in}\) this distinction becomes ambiguous. 


\section{Type System}

Our goal is to introduce type rules that allow us to assign resource costs to lists. Previously, we assigned a resource-pair (\cref{def:resource-pair}) to the resulting type; This is not feasible for lists, as the cost of evaluating an expression depends on the length of the list. \emph{Potentials} are the analogue to recurrence relations, allowing us to assign costs to every list element. 

\subsection{Resource-Aware Lists}\label{sec:resource-aware-lists}

Resource-aware lists have a \emph{potential} assigned to them, which signifies the cost \emph{per list entry}. The potential attributed to a list solely depends on how the list is used; Thus, potentials are of interest in function signatures to signify cost. 

\begin{definition}[Resource-aware list]\label{def:ra-list}
   Let \(p \in \mathbb{N}\), and let \(A\) be a type from \cref{def:type-system-6}. A resource-aware list type is a pair comprising the \emph{base type} and \emph{potential}:

   \[
      \ralist{p}{A} = (\ralist{}{A}, p)
   \]
\end{definition}

\begin{remark}
   Resource demands and potentials can coexist; Resource demands capture static cost for executing an expression. Potentials capture dynamic cost that depends on the length of the list. A function \(f :: A \to \ratype{\ralist{1}{B}}{2}{3}\), for example, indicates two facts (1) Every list entry costs 1 resource and (2) Calling the function requires 2 initial and 3 residual resources, irrespective of the concrete list.
\end{remark}

Given a concrete list that is typed with some potential \(p\), we want to calculate the exact potential residing in the list. There are two constituents: the type parameter \(A \in \type\) and the potential \(p\). The type parameter \(A\) defines how much potential a value \(v : A\) of the list has, whereas the potential \(p\) is the \emph{additional} overhead assigned to every list element - which is specific to the usage of the list.

For any type other than \(A\) within the set of types, we assign a potential of 0. In more intricate type systems that include custom types, it is possible to assign non-zero potentials. 

\begin{definition}[Potential Function]\label{def:potential-function}
   For a value \(v\) of type \(A\), we define the potential of that value as follows:
   \[
      \Phi(v : A) = \begin{cases}
         \Phi(x : B) + \Phi(xs : \ralist{q}{B}) + q   & A = \ralist{q}{B}\\
         0                                            & \text{else}
      \end{cases}
   \]
\end{definition}

To allow for more compact notations of potentials, we introduce \(|l|\) to denote the length of a list \(l\). 

\begin{definition}[Length of lists]\label{def:list-length}
   Given a list \(l : \ralist{}{A}\), we denote by \(|l|\) the \emph{length} of the list.
\end{definition}

As most types \(A \in \type\) have no potential assigned to them, the potential of a list \(l : \ralist{p}{A}\) is dictated \emph{solely} by its length and \(p\); Only if the lists' type parameter is of type \(\ralist{q}{B}\) do values in the list have inherent potential.

\begin{corollary}[Potential of lists]\label{cor:potential-list}
   Let \(l : \ralist{q}{A}\), where \(A \in \type\), s.t. \(\Phi(v : A) = 0\), then:
   \[
      \Phi(l : \ralist{q}{A}) = |l| \cdot q
   \]
\end{corollary}

\begin{proof}
   Using the definition of the potential function, we get:
   \begin{align*}
      \Phi(l : \ralist{p}{A}) &= \underbrace{\sum_{x \in l} \Phi(x : A)}_{=0} + \sum_{x \in l} q \\
                              &= |l| \cdot q
   \end{align*}
\end{proof}

With potentials defined for \(A \in \type\), we naturally extend potentials to contexts. Intuitively, the potential of a context comprises the potentials of every variable in the contexts' domain (see \cref{def:context-domain}). This notion will be central to our new definition of soundness in \cref{sec:soundness-6}.

\begin{definition}[Potential of Contexts]\label{def:context-potential}
   Let \(\Gamma\) be a context.

   \[
      \Phi(\Gamma) = \sum_{x \in \domain{\Gamma}} \Phi(x : \Gamma(x))
   \]
\end{definition}

\subsection{Inference Rules}\label{subsec:type-rules-ListTick}

We start by introducing two rules associated to the two list constructors. An empty list can be constructed freely, with no constraints on its potential. On the other hand, constructing a list from an element and another, compatible, list requires us to propagate the potential. The tail \(xs\) of the list is assumed to have potential \(p\), as result, the list \(\cons{x}{xs}\) also has potential \(p\). 

\[
   \inference[(T:Nil)]
   {}
   {\typing{\Gamma}{nil}{\ralist{p}{A}}}
   \qquad
   \inference[(T:Cons)]
   {}
   {\typing{\Gamma;x : A; xs : \ralist{p}{A}}{cons(x, xs)}{\ratype{\ralist{p}{A}}{p}{0}}}
\]

The potential \(p\) of the list \(l\) is the initial resource cost, specific to destructuring the list into head and tail. The expression \(e_1\) has a resource demand of \((q_0, q_1)\), whereas the evaluation of \(e_2\) has a resource demand of \((q_0 + p, q_1)\). Thus, the initial cost for an element of the list is precisely \(p\). Furthermore, \((q_0, q_1)\) represents a static cost that is paid regardless of the concrete list. We attach \((q_0, q_1)\) to the return type. Similar to (T:Let) from previous chapters, we require that \(l \notin \Gamma\), as this can otherwise lead to undesired side-effects.

\[
   \inference[(T:MatchList)]
   {\typing{\Gamma}{e_1}{\ratype{B}{q_0}{q_1}} \qquad \typing{\Gamma;x : A; xs : \ralist{p}{A}}{e_2}{\ratype{B}{q_0 + p}{q_1}}}
   {\typing{\Gamma;l : \ralist{p}{A}}{\listmatch{l}{e_1}{e_2}}{\ratype{B}{q_0}{q_1}}}
\]

For function application, the input type needs to align with the variables that are applied. The rule (T:Application) allows us to type recursive functions. Most importantly, the type of \(x\) and the input type of \(f\) are then inferred to be the same. For recursive calls on a list, we assume that they recur only on the tail \(xs\). Then, the tail has a potential annotation which is propagated to the function signature, as we assume that the types match.

\[
   \inference[(T:Application)]
   {\typing{\Gamma}{f}{A \to \ratype{B}{q_0}{q_1}}}
   {\typing{\Gamma;x:A}{f x}{\ratype{B}{q_0}{q_1}}}
\]

\begin{remark}
   We denote resource-aware functions just like any other resource-aware type, by adding a resource demand. In the case of functions, this looks like \(A \to \ratype{B}{p_0}{p_1}\). Recall from \cref{def:ra-type} that \(A \to \ratype{B}{p_0}{p_1} = (A \to B, (p_0, p_1))\), where the resource demand represents the cost for evaluating the function.
\end{remark}

The final type rule that we introduce is (T:LetFun). Informally, this rule adds the type of the function \(f\) to the context under which \(e_2\) is evaluated. Of course, \(e_2\) may use \(f\), but only the resource-aware type of \(e_2\) is of interest to us.

\[
   \inference[(T:LetFun)]
   {\typing{\Gamma}{f}{A \to B} \qquad \typing{\Gamma; f : A \to B}{e_2}{\ratype{C}{p_0}{p_1}}}
   {\typing{\Gamma}{\letFun{f}{x}{e_1}{e_2}}{\ratype{C}{p_0}{p_1}}}
\]

\subsection{Example Type Derivations}\label{sec:example-type-derivations}

We introduce examples of varying complexity, to illustrate the nuances of type derivation for functions and lists and acquaint the reader to the newly introduced inference rules.

\begin{example}\label{ex:derivation-add1}
   For our first example, consider the function \(add1\) that increments every integer in a list. The code can be found in \cref{fig:code-add1}. We derive a resource-annotated type stepwise, for sake of illustration. 

   \begin{enumerate}[label=Step \Roman*:]
      \item Application of (T:MatchList) yields two premises. We close the first premise by applying (T:Nil). 

\[
   \inference[(T:MatchList)]
      {
         \inference[(T:Nil)]
         {
         }
         {
            \typing{\Gamma}{l}{\ralist{p}{A}}
         }
         \quad
         \typingHighlight{\typing{\Gamma;x:A;xs:\ralist{p}{A}}{e_1}{\ratype{B}{p_0}{p_1}}}
      }
      {
         \typing{\Gamma;l : \ralist{p}{A}}{\listmatch{l}{\listnil}{e_1}}{B}
      }
\]

For brevity and readability, we denote by \(e_1\) the function body that is executed on a match to cons. Furthermore, we use \(e_2 ~\&~ e_3\) to denote sub-expressions.
\[
   e_1 = \letexp{\_}{\tick{1}}{e_2}
\]

\item Applying (T:Let) splits \(e_1\) up into \(\tick{1}\) and the remaining expression, abbreviated \(e_2\). Applying (T:Tick) closes the left branch of the derivation, allowing us to assign a type to the expression \tick{1}.

\[
   \inference[(T:Let)]
   {
      \inference[(T:Tick)]
      {
      }
      {
         \typing{\Gamma}{\tick{1}}{\ratype{\unit}{1}{0}}
      }
      \quad
      \typingHighlight{\typing{\Gamma;x:A;xs:\ralist{p}{A}}{e_2}{\ratype{B}{q_0}{q_1}}}
   }
   {
      \typing{\Gamma;x:A;xs:\ralist{p}{A}}{\letexp{\_}{\tick{1}}{e_2}}{\ratype{B}{p_0}{p_1}}
   }
\]
\begin{align}
   \text{Where } e_2 = \letexp{x'}{x + 1}{e_3} \nonumber \\
   \constraintHighlight{\text{and } (p_0, p_1) = (1, 0) \sequence (q_0, q_1)} \label{eqn:add1-first-seq}
\end{align}

   \item Again, we apply (T:Let), yielding \(x + 1\) and the sub-expression \(e_3\). Applying the rule (T:Op), we infer that \(x : \typeint\). Thus, \(xs : \ralist{p}{\typeint}\) as well.

\[
   \inference[(T:Let)]
   {
      \inference[(T:Op)]
      {
      }
      {
         \typing{\Gamma;x:\typeint}{x + 1}{\typeint}
      }
      \quad
      \typingHighlight{\typing{\Gamma;x':\typeint;xs:\ralist{p}{\typeint}}{e_3}{\ratype{B}{q_0}{q_1}}}
   }
   {
      \typing{\Gamma;x:A;xs:\ralist{p}{A}}{\letexp{x'}{x + 1}{e_3}}{\ratype{B}{q_0}{q_1}}
   }
\]
\[
   \text{Where } e_3 = \letexp{xs'}{add1~xs}{\cons{x'}{xs'}}
\]

   \item The final application of (T:Let) generates two premises. The first being the recursive function call on \(xs\), and the second being the construction of the final list. We infer that \(B = \ralist{q}{\typeint}\), resulting from the application of (T:Cons), from which the return type is propagated. 

Furthermore, \(q_0 = 0 = q_1\), as neither sub-expressions have any resource demand.

\[
   \hspace*{-2cm}
   \inference[(T:Let)]
   {
      \typingHighlight{\typing{\Gamma;xs:\ralist{p}{\typeint}}{add1~xs}{\ralist{q}{\typeint}}}
      \inference[(T:Cons)]
      {}
      {
         \typing{\Gamma;x':\typeint;xs':\ralist{q}{\typeint}}{\cons{x'}{xs'}}{\ralist{q}{\typeint}}
      }
   }
   {
      \typing{\Gamma;x':\typeint;xs:\ralist{p}{\typeint}}{\letexp{xs'}{add1~xs}{\cons{x'}{xs'}}}{\ralist{q}{\typeint}}
   }
\]
\begin{align}
   \constraintHighlight{\text{where } (q_0, q_1) = (0, 0)} \label{eqn:add1-second-seq}
\end{align}

   \item As a last step, we need to type the recursive function call. The rule (T:Application) allows us to do that. Throughout the previous steps, we derived that the input has type \(\ralist{p}{\typeint}\) and the return type is \(\ralist{q}{\typeint}\). 

\[
   \inference[(T:Application)]
   {
      \typing{\Gamma}{add1}{\ralist{p}{\typeint} \to \ralist{q}{\typeint}}
   }
   {
      \typing{\Gamma;xs:\ralist{p}{\typeint}}{add1~xs}{\ralist{q}{\typeint}}
   }
\]

\item As a last step, we collect numeric constraints, introduced in \cref{eqn:add1-first-seq} and \cref{eqn:add1-second-seq}. With these constraints, we can elucidate a resource demand for the expression \(e_1\). The resource demand then identifies the potentials \(p\) and \(q\) of input and output list, respectively. Combining \cref{eqn:add1-first-seq} and \cref{eqn:add1-second-seq}, we get \((p_0, p_1) = (1, 0)\). Thus, the resulting resource-aware type for \(add1\) is:

\[
   add1 :: \rafunc{\ralist{1}{\typeint}}{\ralist{}{\typeint}}{}{}
\]
\end{enumerate}
\end{example}

\section{Soundness}\label{sec:soundness-6}

The introduction of potentials has a profound effect on the proof of soundness. Previously, the proof centered around inequalities on resource demands - this does not suffice. The \emph{effective cost} that is modelled by type inference is composed of resource demands and the potential of the context. 

This forces us to embellish the inequalities required to hold for soundness. Consider that evaluation and typing judgements have the following form:  \(\evals{E}{e}{v}{p_0}{p_1}\) and \(\typing{\Gamma}{e}{\ratype{A}{q_0}{q_1}}\). In order to define sound resource inequalities which account for potentials, we reconsider initial and residual resources. The amortized \emph{initial} cost of an operation is composed of the concrete initial resources \emph{and} the potential of the context in which the operation is performed. Thus the initial resources are given by \(p_0 + \Phi(\Gamma)\). 

In the case of residual resources, a similar line of thought holds. Here, we additionally account for the potential of the returned value \(v\), which yields \(q_1 + \Phi(v : A)\). With both initial and residual resources defined, we express soundness as a relaxation \((q_0 + \Phi(\Gamma), q_1 + \Phi(v : A)) \relaxation (p_0, p_1)\), yielding two inequalities \(q_0 + \Phi(\Gamma) \geq p_0\) and \((q_0 - q_1) + (\Phi(\Gamma) - \Phi(v : A)) \geq p_0 - p_1\).



\begin{theorem}[Soundness of typing for list-tick language]\label{thm:soundness-7}
   Let \(E\) be an \nameref{def:environment}, \(\Gamma\) be a \nameref{def:context}. Further, let \(A \in \type\), \((p_0, p_1), (q_0, q_1) \in \demand\) and an expression \(e\) that evaluates to the value \(v\). 

   \begin{center}
   If \(\evals{E}{e}{v}{p_0}{p_1}\) and \(\typing{\Gamma}{e}{\ratype{A}{q_0}{q_1}}\), then \(v : \ratype{A}{}{}\) and \((q_0 + \Phi(\Gamma), q_1 + \Phi(v : A)) \relaxation (p_0, p_1)\)
   \end{center}
\end{theorem}

\begin{proof}
   (T:Nil): We know that the expression has the form \(e = \listnil\). Also, \((p_0, p_1) = (0, 0)\) from (E:Nil), and \((q_0, q_1) = (0, 0)\) from (T:Nil). Together with \(\Phi(\listnil : \ralist{p}{A}) = 0\), we can infer the following:
   \begin{align}
      (q_0 + \Phi(\Gamma), q_1 + \Phi(\listnil : \ralist{p}{A})   &= (q_0 + \Phi(\Gamma), q_1)\label{eqn:nil-1} \\
                                                                  &\relaxation (q_0, q_1) \label{eqn:nil-2}\\
                                                                  &= (p_0, p_1)
   \end{align}
   Where \cref{eqn:nil-1} uses \(\Phi(\listnil: \ralist{p}{A}) = 0\), and \cref{eqn:nil-2} uses \cref{lemma:pos-relaxation}. 

   (T:Cons): We know that \(e = \cons{x}{xs}\). We also assume that the tail \(xs\) has some potential \(p\). Applying (T:Cons) gives us \(q_0 = p\) and \(q_1 = 0\). Applying (E:Cons) we get \(p_0 = 0 = p_1\).   
   The change in potential is given by \(\Phi(xs : \ralist{p}{A}) - \Phi(\cons{x}{xs} : \ralist{p}{A})\), as the context \(\Gamma\) comprises \(x\) and \(xs\), where \(x\) has no associated potential. Therefore, the change in potential is precisely \(-p\) - the list got elongated by one element. Thus:

   \[
      \underbrace{q_0 - q_1}_{p} + \underbrace{\Phi(\Gamma) - \Phi(\cons{x}{xs} : \ralist{p}{A}}_{-p}) = 0 = p_0 - p_1
   \] 

   And:

   \[
      q_0 + \Phi(\Gamma) = \Phi(\Gamma) \geq 0 = p_0
   \]

   Which results in \((q_0 + \Phi(\Gamma), q_1 + \Phi(\cons{x}{xs} : \ralist{p}{A}) \relaxation (p_0, p_1)\). 

   (T:MatchList): We know that \(e = \listmatch{l}{e_1}{e_2}\) and there are two possible cases. Either the list is \listnil ~or the list has the form \cons{x}{xs}, which results in (E:MatchNil) or (E:MatchCons) being invoked for the evaluation derivation, respectively. 

   Let us start with the first case, \(l = \listnil\). The type rule yields \(p_0 = 0 = p_1\) and the evaluation rule yields \(q_0 = 0 = q_1\). Furthermore, the context is empty, and the empty list has no potential:

   \begin{align*}
      \Phi(\Gamma) = 0 = \Phi(\listnil : \ralist{p}{A})\\
   \end{align*}

   We apply \cref{lemma:pos-relaxation} in \cref{eqn:match-nil-1} and the fact that \(p_0 = q_0 = 0 = q_1 = p_1\) in \cref{eqn:match-nil-2} to get: 
   
   \begin{align}
      (q_0 + \Phi(\Gamma), q_1 + \Phi(\listnil : \ralist{p}{A})   &\relaxation (q_0, q_1) \label{eqn:match-nil-1}\\
                                                                  &= (p_0, p_1) \label{eqn:match-nil-2}
   \end{align}

   The last case is \(l = \cons{x}{xs}\). This dictates the application of (E:MatchCons) for the evaluation. Given that the list \(l\) has potential \(p\), we get the following assumptions:

   \begin{align}
      p_0            &= 0 \qquad p_1 = 0\nonumber\\
      q_0            &= p \qquad q_1 = 0\nonumber\\
      \Phi(\Gamma)   &= \Phi(x : A) + \Phi(xs : \ralist{p}{A})\label{eqn:match-cons-1}\\
      \Phi(\cons{x}{xs} : \ralist{p}{A}) &= p + \Phi(x : A) + \Phi(xs : \ralist{p}{A})\label{eqn:match-cons-2}
   \end{align}

   With these assumptions, we can prove the desired relaxation. First, using \cref{eqn:match-cons-1} and \cref{eqn:match-cons-2} we get:

   \begin{align}
      p + \underbrace{\Phi(\Gamma) - \Phi(\cons{x}{xs} : \ralist{p}{A})}_{-p} = 0
   \end{align}

   We use \cref{lemma:pos-relaxation} to get the desired relaxation:

   \[
      (q_0 + \Phi(\Gamma), q_1 + \Phi(\cons{x}{xs} : \ralist{p}{A})) \relaxation (p_0, p_1)
   \]

   (T:Application): We know that \(e = f~x\) for some function \(f\) and some variable \(x\). Our hypothesis yields the following assumptions, where the first two rows result from applying (E:Application), and the remaining constraints result from applying (T:Application):

   \begin{align}
      &E(f) = e_f                                      &\evals{\envaugment{E}{y_f}{v_x}}{e_f}{v}{p_0}{p_1} \nonumber\\
      &E(x) = v_x                                      &\evals{E}{f~x}{v}{p_0}{p_1} \nonumber\\
      &\typing{\Gamma}{f}{A \to \ratype{B}{q_0}{q_1}}  &\typing{\Gamma;x : A}{f~x}{\ratype{B}{q_0}{q_1}} \nonumber \\
      &(q_0, q_1 + \Phi(v : B)) \relaxation (p_0, p_1)
   \end{align}

   Using \cref{lemma:pos-relaxation} we get \((q_0 + \Phi(\Gamma), q_1 + \Phi(v : B)) \relaxation (p_0, p_1)\).

   (T:LetFun): We know that the expression \(e\) had the form \(\letFun{f}{x}{e_f}{e_2}\). We assume that \(e_2\) is typed soundly, which gives us three additional assumptions:

   \begin{align*}
      & \typing{\Gamma}{e_2}{\ratype{A}{r_0}{r_1}} & \evals{E}{e_2}{v}{s_0}{s_1} & & (r_0, r_1 + \Phi(v : A)) \relaxation (s_0, s_1) &
   \end{align*}

   Application of (E:LetFun) then gives us that \((p_0, p_1) = (s_0, s_1)\). Additionally, applying (T:LetFun) yields \((q_0, q_1) = (r_0, r_1)\). Using that \((r_0, r_1 + \Phi(v : A)) \relaxation (s_0, s_1)\) from our assumptions, we get that \((q_0, q_1 + \Phi(v : A)) \relaxation (p_0, p_1)\). 

   Using \cref{lemma:pos-relaxation} we get \((q_0 + \Phi(\Gamma), q_1 + \Phi(v : A)) \relaxation (p_0, p_1)\) as desired.

\end{proof}
 
