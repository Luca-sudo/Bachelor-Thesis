\chapter{Introduction}\label{chap:introduction}

Analyzing the performance of algorithms features multiple approaches: worst-case, average-case and amortized analysis. Each of these serves a nuanced purpose, and helps in understanding the overall performance of algorithms better. Amortized analysis, on which automatic amortized resource analysis (AARA) builds, describes the performance of algorithms by analyzing a sequence of instructions. Additionally, instructions can be \emph{amortized} by the state of a data structure. 

Performing amortized analysis by hand, however, is time-consuming and error-prone. We first need to define a suitable potential function; this will allow the state of the data structure to amortize more expensive operations. Afterward we need to calculate the amortized cost of multiple instructions and average their results. AARA aims to automate this process, programmers write code and \emph{additionally} provide minor annotations. The resulting program can be analyzed for performance automatically by a type inference algorithm, allowing programmers to allocate more time to working on software instead of analyzing it. 

There are three key components to automatizing this procedure. First, we introduce a new expression \tick{k}. This expression can be used by programmers to embed a virtual cost into their programs. The \tick{} expression can be used to consume or free virtual resources. Next, we define type rules for our programming language. These type rules are \emph{resource-aware} - they have a cost assigned to them, and combine the cost of subsequent expressions properly. Finally, the program is analyzed using type inference. This step collects all information about resource consumption in the form of \emph{linear constraints}. Solving these constraints then provides an upper-bound on the resource consumption.
