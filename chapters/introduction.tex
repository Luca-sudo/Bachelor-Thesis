\chapter{Introduction}\label{chap:introduction}

Analyzing the performance of algorithms features multiple approaches: worst-case, average-case and amortized analysis. Each of these serves a nuanced purpose, and helps in understanding the overall performance of algorithms better. Amortized analysis, on which automatic amortized resource analysis (AARA) builds, describes the performance of algorithms by analyzing a sequence of instructions. Additionally, instructions can be \emph{amortized} by the state of a data structure. 

Performing amortized analysis by hand, however, is time-consuming and error-prone. We first need to define a suitable potential function; this will allow the state of the data structure to amortize more expensive operations. Afterward we need to calculate the amortized cost of multiple instructions and average their results. AARA aims to automate this process, programmers write code and \emph{additionally} provide minor annotations. The resulting program can be analyzed for performance automatically by a type inference algorithm, allowing programmers to allocate more time to working on software instead of analyzing it. 

There are three key components to automatizing this procedure. First, we introduce a new expression \tick{k}. This expression can be used by programmers to embed a virtual cost into their programs. The \tick{} expression can be used to consume or free virtual resources. Next, we define type rules for our programming language. These type rules are \emph{resource-aware} - they have a cost assigned to them, and propagate the cost of previous expressions properly. Finally, the program is analyzed using type inference. This step collects all information about resource consumption in the form of \emph{linear constraints}, which are collectively solved to provide an upper-bound on the resource consumption.

Let us informally work through the AARA procedure, in order to foster a high-level understanding that is useful for following chapters. First, let us look at example code to see how annotating a program with \textbf{tick} instructions looks like. 

\begin{figure}[H] 
   \begin{minted}{ocaml}
      def add1 l = match l with 
	 | nil -> nil
	 | cons(x, xs) -> let _ = tick 1 in			  
			  let x' = x + 1 in
			  let xs' = add1 xs in
			  cons(x', xs')
   \end{minted}
\end{figure}

In the code above, we define a function \emph{add1}, which given a list of integers returns the list of integers with every element incremented. For example, the function maps \((1, 2, 3, 4)\) to \((2, 3, 4, 5)\). Furthermore, there is a \textbf{tick} expression in line three. This tick instruction resembles a cost of one resource. Thus, incrementing a list costs one resource \emph{per element} in the list.

With this fundamental understanding of the function and its associated cost, let us understand how this cost is embedded into type rules in the form of constraints, and, how to arrive at a resource bound from a set of constraints. 

Types are to programs what sets are to mathematics. In a mathematical setting, an element may be a member of a set. Analogously, a program and its values can have a \emph{type}. Strings and floats are examples of types that most programmers know. In the same way that mathematical functions map between sets, functions, like \(add1\) above, map between types. In this case, the function \(add1\) maps from \(\ralist{}{\typeint}\) to \(\ralist{}{\typeint}\). We denote this by writing \(add1 :: \ralist{}{\typeint} \to \ralist{}{\typeint}\).

Type rules allow us to reason about the type of transformed values. For example, if we know that \(1\) and \(2\) are integers then \(1 + 2\) is also an integer. This kind of reasoning is encoded into type rules in the form of premises and conclusions. The premises have to be fulfilled in order for the conclusion to hold. For the addition of integers, we have "if \(x\) and \(y\) are integers \emph{then} \(x + y\) is also an integer". For values and types we do not use the symbol \(\in\) from set theory. Instead, we write \(x : \typeint\) to denote that \(x\) has type integer. Type rules are written in the following form:

\[
   \inference[(Addition)]
   {x : \typeint \quad y : \typeint}
   {x + y : \typeint}
\]

Back to our example function \(add1\). What could a type rule for the tick expression from the third line look like? Let us first understand what value the tick expression returns, as this will help us understand what type we should assign to \(\tick{1}\). While addition of integers returns an integer, invoking \(\tick{1}\) returns nothing. Readers with experience programming in java might recognize the type \(void\) to typically be used in such settings. We give the type void a different name, we call it \emph{unit}. It is therefore justified to claim that \(\tick{1} : \unit\). 

In order to embed the resource consumption of \tick{1}, we need to first formulate precisely what resources are and how we choose to model them. The most common resources, with respect to which programs are analyzed, are time and space. While we classify both as resources, they differ vastly in their characteristics. While memory can be allocated and subsequently freed, time can only be consumed - it is not possible to generate or free time. If we visualize the evolution of time and space, this may look like the two diagrams below; the left-hand side displays time and the right-hand side displays space.

\begin{align*}
   & \tikzfig{introduction_time} & \quad & \tikzfig{introduction_space} &
\end{align*}

Embedding the entire evolution of an expression into the type rule is to elaborate - we need to reduce our representation. This leads us to model \emph{resource demands} as the resources needed to start evaluating the program \emph{and} the resources remaining afterward. We use tuples as a compact way to denote this. The tuple \((4, 2)\) then expresses that a program requires 4 resources to start evaluating and, once finished, leaves 2 resources remaining.

Thus, \tick{1} from our example above can be described using \((1, 0)\), because \tick{1} consumes 1 resource and leaves non resource free afterward. We can now justify claiming \(\tick{1} : \unit\) with cost \((1, 0)\). The cost can be embedded into the type, to provide a more compact and readable notation, like so: \(\tick{1} : \ratype{\unit}{1}{0}\). We get the following type rule for \tick{1}:

\[
   \inference[(Tick 1)]
   {}
   {\tick{1} : \ratype{\unit}{1}{0}}
\]

There are two limitations, however. First, using this approach we would need to provide a type rule for every \(k \in \mathbb{Z}\). Secondly, we can see that there are many more tuples \((p_0, p_1)\) that allow \tick{1} to be executed. For example, \((2, 1)\) is also valid, because the program can have two resources allocated to it instead of one. 

We address the first limitation by parametrizing the type rule. This forces us to consider two cases for \tick{k}, where \(k \in \mathbb{Z}\). If \(k \geq 0\), \textbf{tick} consumes \(k\) resources, we can simply parametrize our type rule above to conclude \(\tick{k} : \ratype{\unit}{k}{0}\). If \(k < 0 \), now \textbf{tick} \emph{frees} \(k\) resources, we want \(\tick{k} : \ratype{\unit}{0}{k}\). Thus, we could create a type rule for each case respectively.

\begin{align*}
	& \inference[(Tick +)]
	  {k : \typeint \quad k \geq 0}
	  {\tick{k} : \ratype{\unit}{k}{0}}
	&
	\inference[(Tick -)]
	{k : \typeint \quad k < 0}
	{\tick{k} : \ratype{\unit}{0}{k}}
	&
\end{align*}

In chapter 2, we introduce resource demands more thoroughly. One salient concept introduced there is that of a \emph{relaxation} (\cref{def:relaxation}). Informally, if \(a\) is a relaxation of \(b\) then any program that can be executed with resource demand \(b\) can also be executed with resource demand \(a\). Using the idea of relaxations, we can provide more general premise for a type rule for \textbf{tick} - we allow any resource demand that consumes (or frees) \emph{at least} \(k\) resources. This also facilitates merging the two cases above into one.

\[
   \inference[(Tick)]
   {k : \typeint \quad p_0 \geq k + p_1}
   {\tick{k} : \ratype{\unit}{p_0}{p_1}}
\]

Now we can see a linear constraint in the premise, \(p_0 \geq k + p_1\). For an elaborate program, many such constraints will be collected. Solving a set of such constraints, which means finding an optimal solution, is a problem called \emph{linear programming} \cite{LinearProgramming2023}. In practice, the constraints collected by the type inference and passed to an LP-solver, which then returns a solution. 

Returning to our example function we are now able to partially type the third line. It is not clear how we type the remaining parts of the function, and how we can express a cost that respects the length of the input list. A detailed type derivation is provided in \cref{ex:derivation-add1} using the type rules introduced in \cref{subsec:type-rules-ListTick}. There are two key concepts required to derive a resource demand in this setting. First, we need to understand how to \emph{sequence} resource demands. As let expressions allow us to sequentially evaluate expressions, we need to account for prior costs, and combine the prior cost with any newly accrued cost. Secondly, we need to understand how the cost for empty and non-empty lists combine.

When two expressions are sequenced using a let expression, we want their sequential evaluation to have the combined cost of both expressions respectively. Consider an example where we have sequence two expressions with resource demands \((2, 1)\) and \((4, 2)\). We cannot compose them \emph{as is}, because after evaluating the first expression only one resource is left, but the second expression requires four resources. Instead, we have to \emph{relax} (\cref{def:relaxation}) the first resource demand such that we can easily chain both expressions. If we relax \((2, 1)\) into \((5, 4)\) we can compose them, as the four resources left after evaluating the first expression \emph{exactly} match the four resources required by the second expression. Thus, we start with 5 resources, then have 4 resources after which the second expression starts evaluating and returns only one resource. The resulting resource demand for executing both expressions is \((5, 1)\). 

We can visualize this using a simplified form of resource diagrams, which are introduced in \cref{chap:resources}. For a resource demand we draw a line from the resources required to start evaluating, to the resource left after evaluating. This yields the diagram on the left-hand side below. In order to compose these resource demands, we lift the first resource demand which gives us the diagram on the right-hand side.

\[
    \tikzfig{introduction_sequencing_1} \qquad \tikzfig{introduction_sequencing_2}
\]

With an understanding of how resource demands are composed, we can assign a cost to the code from line 3 to 6. Assuming that none of the remaining instructions have a cost associated, we can claim that the branch for non-empty lists (lines 3 to 6) costs 1 resource. We now want to translate the cost of 1 resource in lines 3 - 6 to translate to a cost of 1 resource \emph{for every list element}.


