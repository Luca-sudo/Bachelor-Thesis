\chapter{Linear Amortized Analysis} \label{chap:linear}

In this chapter, we introduce AARA for linear potentials, that is potential functions that are linear with respect to the input parameter. As such, \(f(n) = 3 \cdot n + 2\) is considered a linear potential and \(f(n) = 3 \cdot n^{2} + 2\) is not. 
In chapter \ref{chap:polynomial-potential} we introduce AARA for polynomial potentials, which augments the case of linear potentials by introducing a specific set of polynomial functions as potentials.

\section{Type System} \label{chap:type-system}
In order to bound resource consumption, we need to define a type system that a notion of resources. More precisely, we introduce a type system featuring resource-annotated types. This is done by supplementing types with a potential \(q \in \mathbb{Q}\). The interpretation of the annotated potential $ q \in \mathbb{Q} $ depends on the specific type.
One resource-annotated type is that of a generic list \(L^q(A)\). That is, a list comprising elements of type \(A\), where \emph{every element} of the list has the assigned potential \(q\).
We will see that this induces a potential of the form \(f(n) = q \cdot n\), where \(n\) is the size of the list. 

Another resource-annotated type are functions, written \(A \resourcefunc{p} B\). The meaning of the potentials \(p\) and \(p'\) is different compared to lists. The type \(A \xrightarrow{p/p'} B\) can be interpreted as a function from type A to type B, for which we need \(p\) \emph{additional} resources in order to start the evaluation and are left with \(p'\) resources after evaluation. The resources \(p\) are \emph{additional}, as the type \emph{A} may be resource-annotated itself.

The type system used is given as an EBNF in Figure \ref{fig:type-system} below:

\begin{figure}[H]
\centering
\(A,B = Unit | A \times B | A + B | L^q(A) | A \resourcefunc{p} B\)
\caption{Resource-Annotated Type System}
\label{fig:type-system}
\end{figure}
\todo{Maybe split the type system into primitive two different grammars?}

Besides the aforementioned types, pairs, denoted by \(A \times B\), and sum types, denoted by \(A + B\) are available types. Practical examples for sum types with which the reader might be familiar are, among many: union in C++ and enums in rust. 
Suprisingly, neither pairs nor sum types have a resource-annotation attached to them. \todo{Explain why}.

Before introducing type rules, let us build an intuition for resource-bound types, by working through a rudimentary example. Given the function \emph{addL} in figure \ref{func:add-l} below, we want to calculate an upper bound on heap-space usage. For this, we conclude that a list storing values of a primitive type \emph{A} must allocate two memory cells. One for the value itself, and one for the pointer to the next element in the list. Furthermore, storing a list of type \emph{nil} demands no memory.
This choice is mainly for convenience, as it only alters the resulting amount of memory cells by a constant term.

Equipped with this assumption, we can immediately conclude: Given a list \emph{l} of length \(n\), the function \emph{addL} requires \(2n\) memory cells. Hence, we get \(l : L^{2}(A)\). \emph{addL} requires no additional resources, besides those supplemented by the list \emph{l}. 

Let us now incrementally build up a type for the function \emph{addL}. Because it is a function type, we can start with \(addL: A \resourcefunc{p} B\). The input of \emph{addL} is of type \((int, L^{q}(A))\), a pair comprising an integer and a list. Updating our initial typing, we get \(addL : (int, L^{q}(A)) \xrightarrow{p/p'} B)\). Because \emph{addL} returns a list, we can further update the type \emph{B}, yielding \(addL: (int, L^{q}(A)) \xrightarrow{p/p'} L^{q'}(A)\). Lastly, we need to infer the resource bounds \(p, p', q, q'\). We already inferred that \(q = 2\) and that \(p = p' = 0\). Thus, the only resource annotation missing is \(q'\). Since all the necessary resources are provided by the input list, the resource bound does not increase with respect to the output list. 

Thus, we arrive at the following type for \emph{addL}:

\[\text{addL}: (int, L^{2}(A)) \xrightarrow{0/0} L^{0}(A) \]

\begin{figure}[H]
\centering
\begin{minted}{ocaml}
fun addL i l = match l with | nil -> nil
   | x::xs -> (x + i)::(addL i xs)
end
\end{minted}
\caption{AddL function}
\label{func:add-l}
\end{figure}

In order to automize the above procedure, the rules for inference need to be rigorously defined by means of \emph{type rules}. Furthermore, we need to select a set of potential functions that are specifically handy for automatic analysis. This is the aim of 

\section{The Potential Function}
Before defining the potential function, we need to introduce a couple of definitions in order to permit a rigorous definition.
Let \(A\) be a (resource-annotated) type, denote by \(\llbracket A \rrbracket\) the set of \emph{semantic values} of type A. That is, all the concrete values that belong to type \(A\). \(\llbracket L^{q}(int)\rrbracket \), therefore, describes the set of lists of integers, and \([1;2;3] \in \llbracket L^{q}(int)\rrbracket \). 

When arguing about the potential of a variable, we need to consider the \emph{heap} and the \emph{stack}, denoted by \(H\) and \(V\) respectively. This is because the type of a variables, as well as its potential, can only be inferred if we can map the variable to a concrete value - which is precisely what the stack does. The correct notation would therefore be \(\Phi(V(x) : A)\), which is less ergonomic than writing \(\Phi(x : A)\). For convenience, we use the second notation and assume that a stack \(V\) is given implicitly.
In order to track resource-consumption, we need to supply a heap \(H\). Similarly, we assume the head as implicit and use our ergonomic notation, instead of \(\Phi_{H}(x : A)\).  
We denote the set of types with linear potential by \(\mathcal{A}_{lin}\). 

Throughout this thesis we assume that any primitive types, that is types without a resource-annotation, have no effect on the resource consumption. As such their potential is zero. 
\todo{Elaborate on why this is okay and how to move to non zero potentials}

We now define the potential function for the type system in \ref{fig:type-system}:
\begin{align*}
   \Phi(a : A) &= 0 ~~~ & A &= Unit \\
   \Phi(left(a): A + B) &= \Phi(a: A) \\
   \Phi(right(b): A + B) &= \Phi(b: B) \\
   \Phi((a, b) : A \times B) &= \Phi(a: A) + \Phi(b: B) \\
   \Phi(l : L^q(A)) &= q \cdot n + \displaystyle\sum_{i = 1}^{n} \Phi(a_i : A) & l &= [a_1, \dots, a_n] \\
\end{align*}

As our type system \ref{fig:type-system} permits nesting lists, we could have an element $l$ of type $L^q(L^r(A))$. The potential function we defined also permits those types of values. We get: 
$$\Phi(l : L^q(L^r(A))) = q \cdot n + \displaystyle\sum_{i}^{n} \Phi(l_i : L^r(A)) = q \cdot n + n \cdot (r \cdot m + \Phi(a_i : A)) $$ 
which is a multivariate linear potential, as n and m are the lengths of the respective lists. This illustrates that our definition of linear potential functions allows for multivariate functions.

\subsection{Subtyping and Sharing Relation}
The overarching goal is to automize the elucidation of amortized resource bounds. More specifically, we want to have the tightest bounds possible. In this section, we define the subtyping and sharing relation, which will enable us to elicit the most general resource-annotated type and account for multiple occurences of a resource-annotated variable.

The \emph{Subtyping relation} specifies the conditions under which one (resource-annotated) type is a subtype of another (resource-annotated) type. Intuitively, a resource-annotated type $A$ is a subtype of a resource-annotated type $B$, if they have the same semantic values, i.e. $\llbracket A \rrbracket = \llbracket B \rrbracket$ and the potential of the subtype is greater than the potential of the supertype. Defining subtyping in this way results in the most general resource-annotated typing corresponding to the tightest possible potential.

We define the subtyping relation as the smallest relation that satisfies the following conditions:

\begin{align*}
   A & \subtype A && \text{for an arbitrary type A} \\
   A \times B & \subtype C \times D && \text{iff.} A \subtype C ~ \& B \subtype D \\
   A + B & \subtype C + D && \text{iff.} A \subtype C ~ \& B \subtype D \\
   L^q(A) & \subtype L^r(B) && \text{iff.} A \subtype B ~ \& q \geq r \\
\end{align*}

The \emph{Sharing} relation allows us to have a consistent potential, even if a variable if referenced multiple times. Consider the function ... below as an illustrative example.
\todo{add an example, illustrating sharing}

Intuitively, the sharing relation ensures that whenever there are multiple references to a variable, the potential of the variable that is being referenced is split up into the references. This ensures that our potential function behaves as expected in those cases\dots

We define the sharing relation as the smallest relation, such that the following conditions are satisfied:

\begin{align*}
   A & \shares (A, A) && \text{for an arbitrary type A} \\
   A \times B & \shares ((A_1 \times B_1), (A_2 \times B_2)) && \text{iff.} A \shares (A_1, A_2) ~ \& B \shares (B_1, B_2) \\
   A + B & \shares ((A_1 + B_1), (A_2 + B_2)) && \text{iff.} A \shares (A_1, A_2) ~ \& B \shares (B_1, B_2) \\
   L^q(A) & \shares (L^r(A_1), L^m(A_2)) && \text{iff.} A \shares (A_1, A_2) ~ \& q = r + m \\
\end{align*}
\todo{Make the ampersands prettier}

For the subtyping and sharing relation we have a corrolary that relates them to the potential function we defined earlier.

\begin{lemma}
   \text{Let A and B be types, such that }$A \subtype B.$\text{ It then holds for every} $a~\in~A$ \text{ and } $b \in B$ \text{ that: } $\Phi(a: A) \geq \Phi(b : B).$
\end{lemma}
\begin{lemma}
   \text{Let } $A, A_1, A_2$ \text{ be types, such that }$A \shares A_1, A_2.$ \text{It then holds for every } $a \in A, a_1 \in A_1, a_2 \in A_2$ \text{ that: } $\Phi(a: A) = \Phi(a_1: A_1) + \Phi(a_2 : A_2)$ 
\end{lemma}
\todo{Prettify the Lemmata}




\subsection{Type rules}
\label{sec:type-rules}
