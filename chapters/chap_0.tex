\begin{abstract}
   The performance of programs is of paramount interest to programmers. Automatic amortized resource analysis (AARA) allows the automatic elucidation of resource bounds for involved programs.  

   This thesis presents the work of Hoffmann \cite{hoffmannTypesPotentialPolynomial2011}, with a focus on linear bounds. To achieve a deep understanding, we explore the structure of resource demands which represent the resource consumption of programs. Afterward, we incrementally build up a programming language that features recursion on lists. Each chapter introduces new constructs to our programming language and highlights challenges that arise.  

   Chapter 1 provides a high-level introduction to AARA, introducing fundamental notions from type theory and amortized analysis with a concrete example. 

   Chapter 2 introduces resources together with a sequencing operation and a relaxation relation that facilitates comparing resource demands. 

   Chapter 3 introduces the core fragment of our programming language. With this language we can construct programs that sequentially consume or free resources. 

   Chapter 4 augments the previous programming language with variables. This forces us to introduce environments and contexts in order to associate variables with values (or types). 

   Chapter 5 introduces lists into our programming language. With the introduction of lists we also define potentials which capture cost with respect to lists.
\end{abstract}

\cleardoublepage

\pagestyle{scrplain} % switch off headers and footers
\section*{Official Declaration}
Hereby I declare, that I have not submitted this thesis in this or similar form to any other examination at the Ruhr-Universität Bochum or any other institution or university.

\noindent
I officially ensure, that this paper has been written solely on my own.
I herewith officially ensure, that I have not used any other sources but those stated by me.
Any and every parts of the text which constitute quotes in original wording or in its essence have been explicitly referred by me by using official marking and proper quotation.
This is also valid for used drafts, pictures and similar formats.

\noindent
I also officially ensure that the printed version as submitted by me fully confirms with my digital version.
I agree that the digital version will be used to subject the paper to plagiarism examination.

\noindent
Not this English translation, but only the official version in German is legally binding

\section*{Eidesstattliche Erklärung}
{\selectlanguage{ngerman}
Ich erkläre, dass ich keine Arbeit in gleicher oder ähnlicher Fassung bereits für eine andere Prüfung an der Ruhr-Universität Bochum oder einer anderen Hochschule eingereicht habe.

\noindent
Ich versichere, dass ich diese Arbeit selbstständig verfasst und keine anderen als die angegebenen Quellen benutzt habe. Die Stellen, die anderen Quellen dem Wortlaut oder dem Sinn nach entnommen sind, habe ich unter Angabe der Quellen kenntlich gemacht.
Dies gilt sinngemäß auch für verwendete Zeichnungen, Skizzen, bildliche Darstellungen und dergleichen.

\noindent
Ich versichere auch, dass die von mir eingereichte schriftliche Version mit der digitalen Version übereinstimmt.
Ich erkläre mich damit einverstanden, dass die digitale Version dieser Arbeit zwecks Plagiatsprüfung verwendet wird.\@}

\makeatletter
\vspace{2cm}
\rule{4cm}{0.1pt} \hfill \rule{7cm}{0.1pt} \\
\hspace*{1.5cm} \textsc{Date} \hspace*{7cm} \textsc{\@author}
\makeatother

\tableofcontents

\cleardoublepage
