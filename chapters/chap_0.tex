\begin{abstract}
   The performance of computer programs is of paramount interest to software engineers. There are multiple lenses through which performance may be evaluated: Worst-case analysis, average-case analysis and amortized analysis. In the case of amortized analysis, however, analysis was previously done manually - making it tedious and error-prone. This thesis presents Automatic Amortized Resource Analysis (AARA), a method that permits \emph{automatic} elicitation of a program's resource consumption through augmented type inference.

   AARA embellishes type rules with linear constraints. As part of type inference, those constraints are collected and solved using an LP-solver.

   This thesis presents concepts of AARA by incrementally building up a programming language. In chapter 2 we start off by introducing resource demands, which capture the resource consumption of a program. We then introduce a sequencing operation that allows composing resource demands, to build up larger programs, and a relation that facilitates comparing resource demands.

   In chapter 3 we introduce the \emph{tick} instruction, which is an explicit handle for consuming/freeing resources. In chapter 4, we extend this language by introducing variables. In chapter 5, we introduce lists as a data type. This requires introducing potentials in order to elicit lower bounds on the resource consumption that is variable under the length of the list.
\end{abstract}

\cleardoublepage

\pagestyle{scrplain} % switch off headers and footers
\section*{Official Declaration}
Hereby I declare, that I have not submitted this thesis in this or similar form to any other examination at the Ruhr-Universität Bochum or any other institution or university.

\noindent
I officially ensure, that this paper has been written solely on my own.
I herewith officially ensure, that I have not used any other sources but those stated by me.
Any and every parts of the text which constitute quotes in original wording or in its essence have been explicitly referred by me by using official marking and proper quotation.
This is also valid for used drafts, pictures and similar formats.

\noindent
I also officially ensure that the printed version as submitted by me fully confirms with my digital version.
I agree that the digital version will be used to subject the paper to plagiarism examination.

\noindent
Not this English translation, but only the official version in German is legally binding

\section*{Eidesstattliche Erklärung}
{\selectlanguage{ngerman}
Ich erkläre, dass ich keine Arbeit in gleicher oder ähnlicher Fassung bereits für eine andere Prüfung an der Ruhr-Universität Bochum oder einer anderen Hochschule eingereicht habe.

\noindent
Ich versichere, dass ich diese Arbeit selbstständig verfasst und keine anderen als die angegebenen Quellen benutzt habe. Die Stellen, die anderen Quellen dem Wortlaut oder dem Sinn nach entnommen sind, habe ich unter Angabe der Quellen kenntlich gemacht.
Dies gilt sinngemäß auch für verwendete Zeichnungen, Skizzen, bildliche Darstellungen und dergleichen.

\noindent
Ich versichere auch, dass die von mir eingereichte schriftliche Version mit der digitalen Version übereinstimmt.
Ich erkläre mich damit einverstanden, dass die digitale Version dieser Arbeit zwecks Plagiatsprüfung verwendet wird.\@}

\makeatletter
\vspace{2cm}
\rule{4cm}{0.1pt} \hfill \rule{7cm}{0.1pt} \\
\hspace*{1.5cm} \textsc{Date} \hspace*{7cm} \textsc{\@author}
\makeatother

\tableofcontents

\cleardoublepage
