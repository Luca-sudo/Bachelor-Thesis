\setcounter{tocdepth}{3}  % Inhaltsverzeichnis bis Subsubsection
\setcounter{secnumdepth}{3} % Nummerierung bis Subsubsection

% General stuff
\usepackage[utf8]{inputenc} % CHANGE HERE IF NECESSARY
\usepackage[T1]{fontenc}
\usepackage[english]{babel} % last language given is used (here: english. Alternative: ngerman)
\usepackage{lmodern}
\usepackage{todonotes}
\usepackage{turnstile}
\usepackage{mathtools}
\usepackage{semantic}
\usepackage{multicol}
\usepackage{centernot}
\usepackage{enumitem}
\usepackage{rotating}
\usepackage{pifont}


\usepackage{tikz} % used on frontpage
\usepackage{tikzit}
\usetikzlibrary{decorations.pathreplacing, calligraphy, positioning}
\input{figures/style.tikzstyles}

% TikZ related fixes
\DeclareOldFontCommand{\rm}{\normalfont\rmfamily}{\mathrm}

\usepackage{stmaryrd}


% Set date here
%\day=4 \month=5 \year=2023

% Set name and title
\author{Luca Witt}
\title{Type Systems in Automatic Amortized Resource Analysis}
\date{\today}

% Bibliography
\usepackage[style=numeric,giveninits=false,natbib=true,backend=biber,maxbibnames=99,defernumbers=true,language=autobib,date=long]{biblatex}
\addbibresource{literature/literature.bib}

\usepackage{dsfont}

%%%%%% %%%%%%

% Load packages ...

% Custom packages
\usepackage{amsthm}

% For minted to properly display symbols in lean scripts
\DeclareUnicodeCharacter{25B9}{\sequence}
\DeclareUnicodeCharacter{227D}{\relaxation}
\DeclareUnicodeCharacter{2227}{\ensuremath{\wedge}}

% Custom commands
\newcommand{\shares}{\curlyveedownarrow}
\newcommand{\subtype}{<:}
\newcommand{\resourcefunc}[1]{\xrightarrow{#1/#1'}}
\newcommand{\unit}{\ensuremath{\text{Unit}}}
\newcommand{\injleft}{\text{left}}
\newcommand{\injright}{\text{right}}
\newcommand{\semantic}[1]{\llbracket #1 \rrbracket}
\newcommand{\evaluatesto}[2]{\mathrel{\stackon[1pt]{$\rightsquigarrow$}{\scriptscriptstyle #1 \ #2}}}
\newcommand{\voidenvironment}{\emptyset}

% Command for the evaluation judgement. Takes the environment, expression, cost pair and final value.
\newcommand{\evals}[5]{#1 \vDash #2 \mathrel{\stackon[1pt]{\(\rightsquigarrow\)}{\(\scriptscriptstyle #4 \ #5\)}} #3}

% Resource annotated type. Arguments: type, prior resources, resources freed.
\newcommand{\ratype}[3]{#1 ^{#2}_{#3}}

% Typing judgement. Arguments: Context, expression, type.
\newcommand{\typing}[3]{#1~\vdash~#2:#3}

% Domain of a context
\newcommand{\domain}[1]{\ensuremath{\textbf{dom}(#1})}

% Highlights background color, inidcating that this is examined in the next step
\newcommand{\typingHighlight}[1]{\tikz[baseline]{\node[fill=blue!30, rectangle, anchor=base]{\ensuremath{#1}}}}

\newcommand{\constraintHighlight}[1]{\tikz[baseline]{\node[fill=green!30, rectangle, anchor=base]{\ensuremath{#1}}}}

% Programming language constructs as defined in the respective chapters.
\newcommand{\letexp}[3]{\ensuremath{\mathbf{let}~#1 = #2~\mathbf{in}~#3}}
\newcommand{\tick}[1]{\ensuremath{\mathbf{tick}~#1}}
\newcommand{\true}{\ensuremath{\mathbf{true}}}
\newcommand{\false}{\ensuremath{\mathbf{false}}}

% The set of all variable names.
\newcommand{\Var}{\ensuremath{\text{Var}}}

% The set of all values.
\newcommand{\Vals}{\ensuremath{\text{Vals}}}

% Partial function
\newcommand{\partialto}{\rightharpoonup}

% Add variable to environment
\newcommand{\envaugment}[3]{#1[#2 \mapsto #3]}

% Does an environment contain a variable?
\newcommand{\contains}[2]{#2 \in #1}

% Symbol for type int
\newcommand{\typeint}{\text{Int}}

% Symbol for type bool
\newcommand{\bool}{\text{Bool}}

% Symbol for the set of Types.
\newcommand{\Types}{Type}

% The empty context.
\newcommand{\emptycontext}{\emptyset}

% List type.
\newcommand{\ralist}[2]{\text{List}^{#1}(#2)}

% Set of lists of length n
\newcommand{\listoflength}[2]{\ensuremath{\text{List}_{#1}(#2)}}

% List constructor.
\newcommand{\cons}[2]{#1 :: #2}

% List nil.
\newcommand{\listnil}{\textbf{nil}}

% Pattern match on lists.
\newcommand{\listmatch}[3]{\ensuremath{\textbf{match } #1~\textbf{with}~|~\listnil \to #2~|~\cons{x}{xs} \to #3}}

% Set of resources
\newcommand{\resource}{\ensuremath{\mathcal{R}}}

% Set of resource demands
\newcommand{\demand}{\ensuremath{\mathcal{D}}}

% Sequencing resource demands
\newcommand{\sequence}{\ensuremath{\triangleright}}

% Relaxation of resource
\newcommand{\relaxation}{\ensuremath{\succcurlyeq}}

% Not relaxation of resource
\newcommand{\notrelaxation}{\centernot \relaxation}

% Resource-aware function type
\newcommand{\rafunc}[4]{#1 \to \ratype{#2}{#3}{#4}}

% Equivalent types
\newcommand{\typeequiv}{\equiv}

% Capacity relation
\newcommand{\capacity}{\succ}

% Set of types.
\newcommand{\type}{\ensuremath{\mathbb{T}}}

% Function abstraction
\newcommand{\func}{\ensuremath{\textbf{def}}}

% Lambda expression
\newcommand{\arrow}[2]{\ensuremath{#1 \to #2}}

% Function declaration
\newcommand{\letFun}[4]{\ensuremath{\textbf{letfun}~#1 = #2 \to #3~\textbf{fin}~#4}}

\reservestyle{\command}{\textbf}
\command{let, in, tick}



% Corporate Design
\usepackage{eso-pic}
\usepackage{color}
% Comment out if the RUB fonts are installed
% Link: https://noc.rub.de/~jobsanzl/latex/rubtexfonts-0.4.tar.gz
%\usepackage{rubfonts2009}
\newcommand{\setrubfontnormal}[1]{\fontfamily{rubscala}\fontsize{#1}{1}\selectfont}
\newcommand{\setrubfontextra}[1]{\fontfamily{rubflama}\fontsize{#1}{1}\selectfont}
\definecolor{rubgreen}{cmyk}{0.5,0,1,0}
\definecolor{rubblue}{cmyk}{1,0.5,0,0.6}

% Figures
\usepackage{graphicx}

% Tables
\usepackage{booktabs}
\usepackage{marvosym}
\usepackage{multirow}

% Math stuff and units
\usepackage{amsmath, amssymb, amsfonts}

% Enable quotes by \enquote{}
\usepackage[babel,english=american, german=quotes]{csquotes}

\usepackage{fontawesome5}

\usepackage{minted}
% Listings
\setminted{
	frame=single, % single, lines, leftline, topline, bottomline, none
	fontsize=\footnotesize,
        % fontsize=\scriptsize,
	linenos=true,
	numbersep=5pt,
	breaklines=true,
	breakautoindent=true,
	breakbefore=.,
	breakafter=-,
	escapeinside=~~
}
\BeforeBeginEnvironment{minted}{\smallskip}
\AfterEndEnvironment{minted}{\smallskip}
\setmintedinline{
    fontsize=\small,
    escapeinside=~~,
    breakbefore=.,
    breakafter=-:/\,
}
\newmintinline[code]{text}{}
\newmintinline[js]{javascript}{}
\newmintinline[html]{html}{}
\newmintinline[json]{json}{}

% Format page foot and header
\usepackage{scrlayer-scrpage}
\clearscrheadings
\clearscrheadfoot
\automark[section]{chapter}
\ohead{\pagemark}
\ihead{\headmark}
\pagestyle{scrheadings}

% Hyperlinks and menu for your document
\usepackage[breaklinks,hyperindex,colorlinks,anchorcolor=black,citecolor=black,filecolor=black,linkcolor=black,menucolor=black,urlcolor=black,pdftex]{hyperref} % pagebackref: Add page number to the references where they can be found
\usepackage[capitalize]{cleveref}
\usepackage{nameref}
\usepackage{stackengine}

\theoremstyle{definition}
\newtheorem{theorem}{Theorem}[chapter]
\newtheorem{corollary}[theorem]{Corollary}
\newtheorem{lemma}[theorem]{Lemma}
\newtheorem{definition}[theorem]{Definition}
\newtheorem{example}[theorem]{Example}
\newtheorem*{remark}{Remark}

% Glossary
\usepackage[record, style=alttreegroup, nomain, symbols]{glossaries-extra}

% Symbols. To be used by glossaries package.

% Resources
\glsxtrnewsymbol
  [description={Set of resources (\cref{def:resources})}, group={resources}]
  {resources}
  {\resource}
\glsxtrnewsymbol
[description={Resource Demands (\cref{def:resource-pair})}, group={resources}]
  {resource demands}
  {\demand}
\glsxtrnewsymbol
[description={Sequencing Operator (\cref{def:multiplying-pairs})}, group={resources}]
  {sequencing}
  {\sequence}
\glsxtrnewsymbol
[description={Relaxation Relation (\cref{def:relaxation})}, group={resources}]
  {relaxation}
  {\relaxation}

% Types
\glsxtrnewsymbol
[description={Set of Types (\cref{fig:type-system}, \cref{def:type-system-5}, \cref{def:type-system-6})}, group={types}]
  {set types}
  {\type}
\glsxtrnewsymbol
[description={Unit Type}, group={types}]
  {unit}
  {\(\unit\)}
\glsxtrnewsymbol
[description={The Unit Value}, group={types}]
  {unitval}
  {\(\bullet\)}
\glsxtrnewsymbol[description={Type Bool}, group={types}]{int}{\bool}
\glsxtrnewsymbol[description={Type Int}, group={types}]{bool}{\typeint}
\glsxtrnewsymbol
[description={Resource-Aware Type (\cref{def:ra-type})}, group={types}]
  {ratype}
  {\(\ratype{A}{p_0}{p_1}\)}
\glsxtrnewsymbol[description={Resource-Aware Function}, group={types}]{rafunc}{\(\rafunc{A}{B}{p_0}{p_1}\)}
\glsxtrnewsymbol[description={Context (\cref{def:context})}, group={types}]{context}{\(\Gamma\)}
\glsxtrnewsymbol[description={Domain of a Context (\cref{def:context-domain})}, group={types}]{context-domain}{\(\domain{\Gamma}\)}
\glsxtrnewsymbol[description={Disjoint Contexts (\cref{def:disjoint-contexts})}, group={types}]{disjoint contexts}{\(\Gamma_1 \cap \Gamma_2 = \emptyset\)}
\glsxtrnewsymbol[description={Union of Contexts}, group={types}]{context union}{\(\Gamma_1;\Gamma_2\)}
\glsxtrnewsymbol[description={Augmenting a Context}, group={types}]{context augment}{\(\Gamma;x : A\)}
\glsxtrnewsymbol[description={List Type with Potential (\cref{def:ra-list})}, group={types}]{listtype}{\(\ralist{q}{A}\)}
\glsxtrnewsymbol[description={Set of Lists with length \(n\) (\cref{def:lists-of-length-n})}, group={types}]{set of lists length n}{\listoflength{n}{A}}
\glsxtrnewsymbol[description={Length of List (\cref{def:list-length})}, group={types}]{listlength}{\(|l|\)}
\glsxtrnewsymbol[description={{Potential Function (\cref{def:potential-function})}}, group={types}]{potentialfunction}{\(\Phi\)}
\glsxtrnewsymbol[description={Typing Judgement (\cref{def:type-judgement-context})}, group={types}]{typejudgement}{\(\typing{\Gamma}{e}{A}\)}


% Evaluation
\glsxtrnewsymbol[description={Environment (\cref{def:environment})}, group={evaluation}]{environment}{\(E\)}
\glsxtrnewsymbol[description={Value of x}, group={evaluation}]{value from environment}{\(E(x)\)}
\glsxtrnewsymbol[description={Domain of E (\cref{sec:environments})}, group={evaluation}]{domain of environment}{\domain{E}}
\glsxtrnewsymbol[description={Existence of x in E (\cref{sec:environments})}, group={evaluation}]{def in env}{\(\contains{E}{x}\)}
\glsxtrnewsymbol[description={Map x to y in E (\cref{def:environment-augment})}, group={evaluation}]{envaugment}{\(\envaugment{E}{x}{y}\)}
\glsxtrnewsymbol[description={Evaluation judgement (\cref{def:eval-judgement-environments})}, group={evaluation}]{evaluation}{\(\evals{E}{e}{v}{p_{0}}{p_{1}}\)}
\glsxtrnewsymbol[description={Set of Variable Names (\cref{chap:VarTick})}, group={evaluation}]{var}{\(\Var\)}
\glsxtrnewsymbol[description={Set of Values (\cref{def:vals})}, group={evaluation}]{vals}{\(\Vals\)}

% Prog Language syntax
\glsxtrnewsymbol[description={Let expression (\cref{def:prog-lang-6})}, group={syntax}]{letexp}{\(\letexp{x}{e_1}{e_2}\)}
\glsxtrnewsymbol[description={Tick (\cref{def:prog-lang-6})}, group={syntax}]{tick}{\tick{k}}
\glsxtrnewsymbol[description={List Constructor (\cref{def:prog-lang-6})}, group={syntax}]{cons}{\cons{x}{xs}}
\glsxtrnewsymbol[description={Nil Constructor (\cref{def:prog-lang-6})}, group={syntax}]{nil}{\listnil}
\glsxtrnewsymbol[description={Match on Lists (\cref{def:prog-lang-6})}, group={syntax}]{list match}{\listmatch{l}{e_1}{e_2}}
\glsxtrnewsymbol[description={Function Application (\cref{def:prog-lang-6})}, group={syntax}]{func app}{\(f~x\)}
\glsxtrnewsymbol[description={Arrow/ Lambda(\cref{def:prog-lang-6})}, group={syntax}]{arrow}{\arrow{x}{e_f}}
\glsxtrnewsymbol[description={Function Declaration (\cref{def:prog-lang-6})}, group={syntax}]{func abstraction}{\letFun{f}{x}{e_f}{e_2}}

% Glossary group titles
\renewcommand\glstreegroupheaderfmt[1]{\begingroup\centering \textbf{#1}\par\endgroup}
\glsfindwidesttoplevelname
\glsxtrsetgrouptitle{resources}{Resources}
\glsxtrsetgrouptitle{types}{Typing}
\glsxtrsetgrouptitle{evaluation}{Evaluation}
\glsxtrsetgrouptitle{syntax}{PL Syntax}


% DO NOT LOAD ANY OF YOUR PACKAGES BEYOND THIS PACKAGE

\makeatletter
\AtBeginDocument{
 \hypersetup{
   pdftitle = {\@title},
   pdfauthor = {\@author},
   pdfsubject={\@title},
   pdfkeywords={OpenID Connect, add more}, % CHANGE HERE
 }
}

\definecolor{pairedTwoDarkBlue}{HTML}{1f78b4}
\definecolor{pairedSixDarkRed}{HTML}{e31a1c}
\newcommand{\APm}[1]{\todo[backgroundcolor=blue!10]{\textbf{AP:} #1}}
\newcommand{\AP}[1]{\textbf{\color{pairedSixDarkRed} #1}}
